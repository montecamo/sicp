%
% A note on how to extend the STk interpreter
%
% Copyright � 1993-1999 Erick Gallesio - I3S-CNRS/ESSI <eg@unice.fr>
%
% Permission to use, copy, modify, distribute,and license this
% software and its documentation for any purpose is hereby granted,
% provided that existing copyright notices are retained in all
% copies and that this notice is included verbatim in any
% distributions.  No written agreement, license, or royalty fee is
% required for any of the authorized uses.
% This software is provided ``AS IS'' without express or implied
% warranty.
%
%           Author: Erick Gallesio [eg@unice.fr]
%    Creation date:  in 1993
% Last file update:  3-Sep-1999 19:36 (eg)
%

\documentclass[10pt]{article}
\usepackage{a4wide}
\usepackage{fancyheadings}
\usepackage{fancybox}
\usepackage{eg-commands}

\pagestyle{fancyplain}
\makeindex\parindent0pt\parskip2mm
\begin{document}
\bibliographystyle{plain}

\title{Extending the ST{\large\bf{K}} interpreter}
\author{Erick Gallesio \\
Universit\'e de Nice~~-~~Sophia-Antipolis \\
Laboratoire I3S - CNRS URA 1376 - ESSI. \\
Route des Colles\\
B.P. 145\\
06903 Sophia-Antipolis Cedex - FRANCE\\[3mm]
email: eg@unice.fr}
\date{July 1995}
\maketitle

\begin{abstract}
This document describes how to extend the {\stk} interpreter with new
primitives procedures and/or new types. Extending the interpreter can be done
by writing new {\em modules} in C. New C code can be statically linked to the
core interpreter or dynamically loaded on operating systems which support
shared libraries. This document also presents how to integrate new
Tk widgets written for the Tcl interpreter in {\stk}.
\end{abstract}


\pagebreak
\tableofcontents
\pagebreak
\section{Introduction}
This document describes how to extend the {\stk}\cite{Gallesio93-1}
interpreter using the C language\cite{Kernighan:CPL88}. To begin, we
will start with a simple extension which will only consist to add some
simple new primitives to the interpreter. Second section will describe
how to add a new type (and the primitives for manipulating this new
type). Another interesting extension consists to add new kind of
primitives (i.e.  primitives which evaluate their argument in
particular way). This kind of extension will be discussed in the third
section.  Fourth section discusses how to add a new widget to the
interpreter. Calling some Scheme code from a C function is showed in
section 5. And last, we will show how to load an extension at load
time. This facility will permit to extend the {\stk} interpreter
without having to recompile it, on systems which support dynamic
loading.

\section{Adding new primitives}

\subsection{A simple example}

\label{simple-example}
One of the simpler extension one can wish to do consists to add new
primitives procedures to the interpreter. To illustrate this, suppose
we want to add two new primitives to the interpreter: {\tt posix-time}
and {\tt posix-ctime}.  The former function correspond to the
POSIX.1\cite{POSIX.1-90} function {\tt time}: it returns the number
of seconds elapsed since 00:00:00 on January 1, 1970, Coordinated
Universal Time (UTC).  The latter is a wrapper around the POSIX.1
function {\tt ctime} which returns a string containing the current
time in an human readable format.

First, we will see how to write the new Scheme primitive {\tt posix-time}.
Implementing a new primitive requires to write a new C function which will do
the work. Here, we write the C function {\tt posix\_time} to implement the
Scheme primitive {\tt posix-time}. The code of this function is given below.
\begin{Code}
\begin{listing}[200]{2}
static PRIMITIVE posix_time(void)
{
  return STk_makeinteger((long) time(NULL));
}
\end{listing}
\end{Code}
This function uses the interpreter \Indextt{STk\_makeinteger} function
which converts a C long integer to a {\stk} integer. Once the {\tt
posix\_time} C function is written, we have to bind this new primitive to
the Scheme symbol {\tt posix-time}. This is achieved by the following C
function call.
\begin{Code}
\begin{listing}[200]{2}
STk_add_new_primitive("posix-time", tc_subr_0, posix_time);
\end{listing}
\end{Code}

\paragraph*{Note:} The C type \Indextt{SCM} is used to describe the objects
manipulated in Scheme. \Indextt{PRIMITIVE} is an alias for this type; it
is preferably used when defining a new primitive.

\Indextt{STk\_add\_new\_primitive} tells the interpreter that the Scheme 
symbol {\tt posix-time} must be bound to the (C written) primitive
{\tt posix\_time}.  The constant \Indextt{tc\_subr\_0} used as the second
argument indicates the arity of this primitive. In this case, the
arity of the primitive is 0.

Let's now have a look at the primitive {\tt posix-ctime}. A first
writing of this primitive could be
\begin{Code}
\begin{listing}[200]{2}
static PRIMITIVE posix_ctime(void)
{
  char *s;
  time_t t = time(NULL);

  s = ctime(&t);
  return STk_makestring(s);
}
\end{listing}
\end{Code}

This functions uses another interpreter routine (\Indextt{STk\_makestring})
which takes as parameter a null terminated string and returns a
Scheme string. 

Binding of the scheme symbol {\tt time-string} to the C function 
{\tt get\_time} is done by the call 
\begin{Code}
\begin{listing}[200]{2}
STk_add_new_primitive("posix-ctime", tc_subr_0, posix_ctime);
\end{listing}
\end{Code}

A complete listing of this code is given in Figure~\ref{posix-1}.
Provided that we have done a shared object of this file, and that its
name is {\tt posix.so}, our two new primitives can be loaded
dynamically by:
\begin{Code}
\begin{listing}[200]{2}
(load "time.so")
\end{listing}
\end{Code}

\paragraph*{Notes:}
\begin{itemize}
\item Suffix can be omitted. Suffixes given in the Scheme variable 
\Indextt{*load-suffixes*} gives the order in which suffixes must be
tried for loading a file.  Default value for this variable is
{\tt ("stk" "stklos" "scm" "so")}.

\item When dynamic loading is used, the interpreter try to call a 
function whose name is equal to the string {\tt "STk\_init\_"} followed by the
name of the file, without suffix. Definitions of new primitives are generally done
in this function. Here, the C function in charge of module initialization must
be called {\tt STk\_init\_posix}.
\end{itemize}

%%%%%%%%%%%%%%%%%%%%%%%%%%%%%%%%%%%%%%%%%%%%%%%%%%%%%%%%%%%%%%%%%%%%%%%%%%%%%%
\begin{figure}
\begin{quote} \footnotesize
\begin{alltt}
#include <sys/types.h>
#include <sys/time.h>
#include <time.h>
#include <stk.h>        {\em /* Declaration of STk objects/primitives */}

static PRIMITIVE posix_time(void)
\{
  return STk_makeinteger((long) time(NULL));
\}

static PRIMITIVE posix_ctime(void)
\{
  char *s;
  time_t t = time(NULL);

  s = ctime(&t);
  return STk_makestring(s);
\}

PRIMITIVE STk_init_posix(void)
\{
  STk_add_new_primitive("posix-time",  tc_subr_0, posix_time);
  STk_add_new_primitive("posix-ctime", tc_subr_0, posix_ctime);
  return UNDEFINED;
\}
\end{alltt}
{\caption{A first version of file {\tt posix.c}}}
\label{posix-1}
\vskip2mm\hrule\vskip3mm
\end{quote}
\end{figure} 

\subsection{Passing arguments to a primitive}

This section shows how to pass arguments to a new primitive written in
C. To illustrate our purpose, we will rewrite the primitive {\tt
posix-ctime} to be conform to POSIX.1 (this function should take an
integer, a count of seconds, and should return corresponding date as a
string). A second writing of previous function could be:

\begin{Code}
\begin{listing}[200]{2}
static PRIMITIVE posix_ctime(SCM seconds)
{
  long sec;

  sec = STk_integer_value_no_overflow(seconds);
  return STk_makestring(ctime((time_t *)&sec));
}
\end{listing}
\end{Code}


This function has one parameter since Scheme primitive arity is one. The
C primitives parameters are always \Indextt{SCM} objects. An object of this type is
a pointer to a \Indextt{struct obj}: the type which permits to represent all the
Scheme objects.  The {\tt SCM} and {\tt struct~obj} types definitions can
be found in the \Indextt{Src/stk.h} header file. 

The first job of this function consists to convert the Scheme
parameter ({\tt seconds}) to a C integer {\tt long int}. This is done with the
function \Indextt{STk\_integer\_value\_no\_overflow}which takes a {\tt SCM}
and returns a {\tt long int}. This functions returns {\tt LONG\_MIN} if
the argument is not a an integer number (or a number which don't fit in the C
representation of a C {\tt long int}). Once this conversion is done,
the rest of the job is similar to the code presented above.

To add this primitive to the global Scheme environment, we have to
change the previous {\tt STk\_add\_new\_primitive} for this primitive by:

\begin{Code}
\begin{listing}[200]{2}
STk_add_new_primitive("posix-ctime", tc_subr_1, posix_ctime);
\end{listing}
\end{Code}

in the init section. This call states that the type of this primitive
is fixed to a \Indextt{tc\_subr\_1} (a arity-1 primitive).

However, this function is not too satisfying, even if close to the
POSIX definition: it obliges to pass a parameter which will be
probably most of the time the result of the primitive {\tt
posix-time} (i.e. the most frequent usage of this function will be

\begin{Code}
\begin{listing}[200]{2}
(posix-ctime (posix-time))
\end{listing}
\end{Code}

which is not very elegant). A better approach consists to allow this
primitive to have a optional parameter. This permits to be at the
same time conform to the POSIX convention and close to Scheme habits. 
The following version implements the {\tt posix-ctime} with a optional
parameter:
\begin{Code}
\begin{listing}[200]{2}
static PRIMITIVE posix_ctime(SCM seconds)
{
  long sec;

  sec = (seconds==UNBOUND) ? time(NULL)
                           : STk_integer_value_no_overflow(seconds);
  return STk_makestring(ctime((time_t *) &sec));
}
\end{listing}
\end{Code}

If the Scheme {\tt posix-ctime} primitive is called with one parameter, it
will be passed to the C function in the {\tt seconds} parameter. If {\tt
posix-ctime} is called without parameter, {\tt seconds} is set to the special
value \Indextt{UNBOUND}. So, the first test in this function consists to set a
correct value to the variable {\tt sec}; this value is either the current
time, either the given integer, depending of the number of parameters
given to {\tt posix-ctime}.

Of course, the type of this new primitive must be changed to allow 0 or 1
parameter. This is done by changing the {\tt tc\_subr\_0} in the previous 
{\tt STk\_add\_new\_primitive} by \Indextt{tc\_subr\_0\_or\_1}.

The following types are available for C primitives:
\begin{itemize}
\item \Indextt{tc\_subr\_0} for arity-0 primitives
\item \Indextt{tc\_subr\_1} for arity-1 primitives
\item \Indextt{tc\_subr\_2} for arity-2 primitives
\item \Indextt{tc\_subr\_3} for arity-3 primitives
\item \Indextt{tc\_subr\_0\_or\_1} for primitives 
      which have 0 or 1 parameter
      (e.g.~{\tt read}). On the C side you have to declare a function
      which takes one {\tt SCM} argument. This argument is set to the
      (evaluated) parameter if present, to {\tt UNBOUND} otherwise.
\item \Indextt{tc\_subr\_1\_or\_2} for 
      primitives which have 1 or 2 parameters
      (e.g.~{\tt write}). Here you have to declare a C function with
      two {\tt SCM} parameters. The first one will contain the first Scheme 
      argument and the second will contain the second argument value
      or {\tt UNBOUND} if omitted.
\item \Indextt{tc\_subr\_2\_or\_3} for 
      primitives which have 2 or 3 parameters
      (there's no primitive of this type in core interpreter).
      Of course, you'll have to declare a C function with three {\tt SCM}
      parameters. Apart that, conventions are the same has before.
\item \Indextt{tc\_lsubr} for primitives which have a variable number of
      arguments. Actuals arguments are collected in a list which is given as the
      first argument of the C primitive. The second argument of the C function
      is an integer counting the actual number of arguments given to the
      primitive. Hence, the signature of the C function which implement a {\tt
      tc\_lsubr} must be
\begin{quote}
\begin{verbatim}
PRIMITIVE function(SCM arglist, int argcount);
\end{verbatim}
\end{quote}
      Note that all the Scheme arguments are evaluated during the
      construction of the list which is passed to the C function.
\item \Indextt{tc\_fsubr} is similar to {\tt tc\_lsubr} except that
      arguments are not evaluated. On the C side, you have to declare a function
      with three {\tt SCM} parameters: the list of (non evaluated) arguments,
      the current environment and the length of the arguments list. The
      signature of the C function which implement a {\tt tc\_fsubr} must be
\begin{quote}
\begin{verbatim}
PRIMITIVE function(SCM arglist, SCM env, int argcount);
\end{verbatim}
\end{quote}
      See \ref{evaluating-args} more details about {\tt tc\_fsubr}.

\item {\tt tc\_tkcommand} for primitives which follow  the Tcl
      command argument passing style (i.e.~{\em \`a la} {\tt argc/argv}). This
      is this kind of procedure that will be used for to add new widgets in the
      {\stk} interpreter. See \ref{new-widget} and \cite{ouster-book} for more
      details.
\end{itemize}

To illustrate how to write a {\tt tc\_lsubr} primitive, let's have a look at the
code, given below, of the function which implement the Scheme primitive {\tt
vector}:
\begin{quote}{\small
\begin{verbatim}
PRIMITIVE STk_vector(SCM arglist, int argcount)
{
  int j;
  SCM z = STk_makevect(argcount, NULL);
  
  for (j = 0; j < argcount; j++, arglist=CDR(arglist)) {
    VECT(z)[j] = CAR(arglist);
  }
  return z;
}
\end{verbatim}
}\end{quote}

This function receives the values passed to the {\tt vector} primitives in the
list arglist (the length of this list is stored in {\tt argcount}). This
function uses \Indextt{STk\_makevect} which returns a Scheme vector. Its first
argument is the length of the vector and its second argument is the initial
value of the vector's elements.  Next section will show how to implement a
primitive which evaluates itself its parameters (i.e. a {\tt tc\_fsubr}
primitive.

\subsection{Evaluating arguments}

In some circumstances it could be useful to add new primitives which don't
evaluate their arguments. This permits to add new control structures to the
interpreter. To illustrate this, we will add two new primitives to the {\stk}
interpreter: {\tt when} and {\tt unless}. As explained in the preceding
section, the C functions which will implement those control structures must be
of type \Indextt{tc\_fsubr}. A {\tt tc\_fsubr} primitive, on the C
side, is given three parameters when called: 
\begin{enumerate}
\item a list of its (non evaluated) parameters, 
\item the local environment when it was called (and in which
evaluations should generally take place),
\item the length of the parameters list.
\end{enumerate}

The C function can step through its parameter list using the C macros
{\tt CAR}, {\tt CDR} and {\tt NULLP} (which do the obvious work) and evaluates elements
of this list as needed. Evaluation of an expression can be done with
the \Indextt{STk\_eval} C function. {\tt STk\_eval} takes two parameters: the
expression to evaluate and the environment in which evaluation takes
place (the \Indextt{NIL} variable, by convention, denotes the Index{global environment}).

\paragraph*{Note:} {\bf a list of arguments is always a proper list. You don't need to test
if it is well formed.}

Hereafter is the code of the {\tt when} primitive.
\begin{Code}
\begin{listing}[200]{2}
static PRIMITIVE when(SCM l, SCM env, int argcount)
{
  SCM res = UNDEFINED;

  if (argcount > 1) {
    if (STk_eval(CAR(l), env) != Ntruth) {
      for (l = CDR(l); !NULLP(l); l = CDR(l)) {
        res = STk_eval(CAR(l), env);
      }
    }
  }
  return res;
}
\end{listing}
\end{Code}

\noindent
Some points to note here:
\begin{itemize}
\item \Indextt{UNDEFINED} is an interpreter {\em constant}. It serves to denote
the notion of {\em ``\Index{unspecified result}''} of {\rrrr}.
\item \Indextt{Truth} and \Indextt{Ntruth} are two global {\em constants} of the
interpreter which denote respectively the \Indextt{\#t} and \Indextt{\#f} Scheme
constants.
\end{itemize}

\noindent
Figure~\ref{when} shows a complete implementation of {\tt when} and {\tt unless}.

\begin{figure}
\begin{quote}\footnotesize
\begin{verbatim}
#include <stk.h>

static PRIMITIVE when(SCM l, SCM env, int argcount)
{
  SCM res = UNDEFINED;

  if (argcount > 1) {
    if (STk_eval(CAR(l), env) != Ntruth) {
      for (l = CDR(l); !NULLP(l); l = CDR(l)) {
        res = STk_eval(CAR(l), env);
      }
    }
  }
  return res;
}

static PRIMITIVE unless(SCM l, SCM env, int argcount)
{
  SCM res = UNDEFINED;

  if (argcount > 1) {
    if (STk_eval(CAR(l), env) == Ntruth) {
      for (l = CDR(l); !NULLP(l); l = CDR(l)) {
        res = STk_eval(CAR(l), env);
      }
    }
  }
  return res;
}

PRIMITIVE STk_init_when_unless(void)
{
  add_new_primitive("when",        tc_fsubr, when);
  add_new_primitive("unless",      tc_fsubr, unless);
  return UNDEFINED;
}
\end{verbatim}
\end{quote}
{\caption{Source listing of file {\tt when\_unless.c}}}
\label{when}
\vskip2mm\hrule\vskip3mm
\end{figure}

\subsection{Signaling errors}
For now, only one function is provided to signal errors: \Indextt{STk\_err}. This
function takes two parameters: a C string which constitutes the body of the
message and a Scheme object (a {\tt SCM} pointer) designating the {\em
erroneous} object.  If the second argument is \Indextt{NIL}, it will not be
printed. Execution of the function {\tt STk\_err} never returns.  It provokes a
jump at the start of the top-level loop. Hereafter, is a new implementation of
the {\tt when} function which uses {\tt STk\_err} when given an erroneous
arguments list.
\begin{Code}
\begin{alltt}
static PRIMITIVE when(SCM l, SCM env, int argcount)
\{
  SCM res = UNDEFINED;

  switch (argcount) \{
    case 0:  STk_err("when: no argument list given", NIL);
    case 1:  STk_err("when: null body", NIL);
    default: {\em /* Argument list is well formed.
              * Evaluate each expression of the body 
              */}
             if (STk_eval(CAR(l), env) != Ntruth)
               for (l = CDR(l); !NULLP(l); l = CDR(l)) 
                 res = STk_eval(CAR(l), env);
  \}
  return res;
\}
\end{alltt}
\end{Code}

\section{Variables}

This section shows how you can access a Scheme variable within C code. It also
shows how you can connect a Scheme and C variable such that modifying it in
Scheme will modify the associated variable and {\em vice versa}.

\subsection{Scheme Symbols and Variables}

\begin{Lentry}
\item[Defining a symbol]
Interning a symbol in the global table of symbols is done with the
\Indextt{STk\_intern} C function. Since this function is often used, you can use
the C macro \Indextt{Intern} as a shortcut. The result of Intern is the {\tt SCM}
object which denotes the scheme symbol associated to the C string passed as parameter.
For example, assigning the list 
\begin{Code}
\begin{listing}[200]{2}
'(green orange red)
\end{listing}
\end{Code}
to the C variable {\tt fire} can be done by
\begin{Code}
\begin{listing}[200]{2}
SCM fire =  Cons(Intern("green"),
                 Cons(Intern("orange"),
                      Cons(Intern("red"), NIL)));
\end{listing}
\end{Code}
Since this notation is difficult to read, some macros have been defined in
\Indextt{Src/stk.h} for building list. These macros are called \Indextt{LISTx} where
x is a number (comprised between 1 and 9) which represent the length of the list
to create. Thus, the previous example could have been written as 
\begin{Code}
\begin{listing}[200]{2}
SCM fire =  LIST3(Intern("green"), Intern("orange"), Intern("red"));
\end{listing}
\end{Code}

\item[Reading a variable]
Reading a variable in Scheme corresponds in fact to look at the value associated
to a symbol. The value associated to a symbol can be obtained with the 
\Indextt{STk\_get\_symbol\_value} C macro. This macro returns a {\tt SCM} object
which correspond to the value associated to the symbol whose name is equal to
the parameter string. \Indextt{STk\_get\_symbol\_value} returns the special
value \Indextt{UNBOUND} is this symbols has no value in the global
environment. The following piece of code
\begin{Code}
\begin{listing}[200]{2}
{
  SCM val = STk_get_symbol_value("foo");

  if (val == UNBOUND)
    STk_err("foo is undefined", NIL);
  else
    STk_display(val, UNBOUND);
}
\end{listing}
\end{Code}
displays the value of the {\tt foo} symbol, or a message is {\tt foo} is undefined in the
global environment. Note the use of the \Indextt{STk\_display} function which
implement the behavior of the Scheme {\tt display} primitive. This call correspond to
a call to {\tt display} with only one parameter, since second parameter is set
to {\tt UNBOUND} (output is done on the standard output port in this case).

\item[Setting a variable]
Setting a Scheme variable corresponds to associate a new value to a symbol. The
value of a symbol can be set with the \Indextt{STk\_set\_symbol\_value} C macro.
For example, 
\begin{Code}
\begin{listing}[200]{2}
STk_set_symbol_value("bar", STk_makeinteger(3L));
\end{listing}
\end{Code}
sets the value of the {\tt bar} symbol to the integer 3. Note that you can set a
symbol in C without using a {\tt define} form as it is necessary in Scheme.
\end{Lentry}

\subsection{Connecting Scheme and C variables}

When building a specialized interpreter, it could be useful to have a variable
you can access both in Scheme an in C. Modifying such a variable in C must
modify the Scheme associated variable and, symmetrically, modifying it in Scheme
must modify the corresponding C variable. One way to do this connection consists
to create a special Scheme variable whose content is read/written by a special
getter/setter. Definition of such a variable, is done by calling the function
\Indextt{STk\_define\_C\_variable}. The C prototype for this function is
\begin{Code}
\begin{listing}[200]{2}
void STk_define_C_variable(char *var, 
                           SCM  (*getter)(char *var), 
                           void (*setter)(char *var, SCM value));
\end{listing}
\end{Code}

The following piece of code shows how we can connect the Scheme variable {\tt *errno*} to the C
variable {\tt errno}:

\begin{Code}
\begin{listing}[200]{2}
static SCM get_errno(char *s)
{
  return STk_makeinteger((long) errno);
}
\end{listing}
\end{Code}
\begin{Code}
\begin{listing}[200]{2}
static void set_errno(char *s, SCM value)
{
  long n = STk_integer_value_no_overflow(value);

  if (n == LONG_MIN) Err("setting *errno*: bad integer", value);
  errno = n;
}
\end{listing}
\end{Code}
\begin{Code}
\begin{listing}[200]{2}
{
  ...
  STk_define_C_variable("*errno*", get_errno, set_errno);
  ...
}
\end{listing}
\end{Code}

After this call to {\tt STk\_define\_C\_variable}, reading ({\em
resp}. writing) the value of the {\tt *errno*} Scheme variable calls the
{\tt get\_errno} ({\em resp}. {\tt set\_errno}) C function.

\section{Calling Scheme from C}

Sometimes, it could be necessary to execute some Scheme code from a C function.
If the Scheme function you have to call is a primitive, it is preferred to call
directly the C function which implement it. To know the name of the C function
which implement a Scheme primitive, you'll have to look in the C file {\tt
primitive\.c} which contains the list of all the primitives of the core
interpreter. If the Scheme code you want to execute is not a call to a
primitive, it is generally easier to put your code in a C string and call the C
function \Indextt{STk\_eval\_C\_string}. This function takes two parameters: the string
to evaluate and the environment in which evaluation must take place. As for {\tt
STk\_eval}, a \Indextt{NIL} value for the environment denotes the global
environment. Suppose, for instance, that you have already written in Scheme the
{\tt fact} procedure; evaluating the factorial of 10 can be done in C with:
\begin{Code}
\begin{listing}[200]{2}
STk_eval_string("(fact 10)", NIL);
\end{listing}
\end{Code}
This call returns a pointer on a Scheme object (a {\tt SCM} pointer) containing
the result of the evaluation. If an error occurs during evaluation. It is
signaled to the user and the constant NULL is returned by {\tt STk\_eval\_string}.

\section{Adding new types}

This sections discusses how to add a new type to the {\stk} interpreter. Interested
reader can find some new types definitions in the {\tt Extensions} directory of
STk. {\stklos}, in particular, is written as an extended type whose definition
is dynamically done as soon as {\em objects} are needed. Hash tables, processes
and sockets are other examples of extended types. 

\subsection{Definition of a Scheme extended type}

Defining a scheme extended type is a little bit more complicated than
defining new primitives since it implies to take into account how this
new type interact with the GC (\Index{Garbage Collector}). Note that
until now we have not discussed about GC problems since the
interpreter is able to hide you it, as far as you don't define new
types. 

To illustrate the discussion, we will show how to add the {\em stack}
type to the {\stk} interpreter in this section. The complete code for
this section can be found in appendix.\label{stack}

\subsubsection{How the GC works}

Before showing how to define a new Scheme type, it is important to understand
how the GC works. First a certain number of cells are created\footnote{by
default 20~000; Use the -cells option of the interpreter to change this
default}. When the interpreter needs a new cell, in the {\tt cons} primitive for
instance, it will take an unused cell in the pool of pre-allocated cells. If no
more cell is available in this area, the GC is called. Its works is divided in
two phases. First phase consists to mark all the cells which are currently in
use. Finding the cells which are in used is done by marking recursively all the
object which are accessible from
\begin{itemize}
    \item the Scheme symbol table,
    \item the registers used by the program,  
    \item the C stack,
    \item global variables of type {\tt SCM}.
\end{itemize}
Marking phase is recursive; that means that if a variable denotes a list, all
the elements of this list have to be marked , to avoid that the GC frees some of
them. Of course, the recursive call for marking the component of a cell depends
on the cell's type. This first phase is called the {\em \Index{marking phase}}.

The second phase of the GC is called the {\em \Index{sweeping phase}}. It is
relatively simple: each allocated cells whose mark bit is unset is placed in the
list of free cells, since nobody points anymore on it.

If no cells can be obtained when the sweeping phase terminates, the pool of
pre-allocated cells will be extended by a new bank of cells. 

\subsubsection{The Extended type data structure}

Defining a new Scheme type consists mainly to define a new
\Indextt{STk\_extended\_scheme\_type} structure and fill in this fields. This
structure is defined as:
\begin{Code}
\begin{listing}[200]{2}
typedef struct {
  char *type_name;
  int  flags;                   
  void (*gc_mark_fct)(SCM x);
  void (*gc_sweep_fct)(SCM x);
  SCM  (*apply_fct)(SCM x, SCM args, SCM env);
  void (*display_fct)(SCM x, SCM port, int mode);
} STk_extended_scheme_type;
\end{listing}
\end{Code}
Each field of this structure is defined below
\begin{Lentry}
\item[type\_name] is a string. It denotes the external name of the new type. The
purpose of this field is mainly for debugging.

\item[flags] is the union of binary constants. For now, only two constants are
defined:
    \begin{itemize}
    \item \Indextt{EXT\_ISPROC} must be set if the new type is a procedure (i.e. if the Scheme
    procedure must answer \#t when called with this object).
    \item \Indextt{EXT\_EVALPARAM} must be set if the new type must evaluates its parameters
    when used as a function.
    \end{itemize}

\item [gc\_mark\_fct] is a pointer to the function which marks objects of the
extended type. The code associated to this function is simple. It consists to
call \index{STk\_gc\_mark} on each field whose type is {\tt SCM} in the type
associated data. This function is automatically called by the interpreter when
it scans all the used cells in the GC \Index{marking phase}.
One example of {\tt gc\_mark\_fct} is given below. 

\item [gc\_sweep\_fct] is a pointer to the function which frees the resources
allocated for representing the new type of object. This function is
automatically called by the GC in the sweeping phase for each cell which is
unused. 

\item [apply\_function] is a pointer to a function which is called when applying
this object to a list of arguments.  This function can only be called if the bit
\Indextt{EXT\_ISPROC} is set. The arguments given to this function are 
evaluated if the \Indextt{EXT\_EVALPARAM} bit is set. Finally, the environment in
which the call is done is passed as the third argument of the {\tt
apply\_function}. It serves principally when the {\tt EXT\_EVALPARAM} bit is
unset.

Set the {\tt apply\_function} to NULL to use the interpreter default apply
function. The default function raises an error when called. You can use the
default function when the new type you define is not a function.

\item [display\_fct] is a pointer to a C function which displays objects of the
new type. The display function has three parameters. The first parameter is the
object to print. The second parameter is the port to which the object must be
printed. Printing an object must be done with one of the following functions
{\tt
\begin{itemize}
\item int  \Index{STk\_getc}(FILE *f);
\item int  \Index{STk\_ungetc}(int c, FILE *f);
\item int  \Index{STk\_putc}(int c, FILE *f);
\item int  \Index{STk\_puts}(char *s, FILE *f);
\item int  \Index{STk\_eof}(FILE *f);
\end{itemize}
}
Those functions are extensions of their C equivalent: they are able to handle the
{\stk} string ports.

The third parameter of the {\tt display\_fct} is a mode constant which can take three
different values. 
\begin{itemize}
\item \Indextt{DSP\_MODE} is used when the object must be {\em displayed} in a
human readable format (as with {\tt display});
\item \Indextt{WRT\_MODE} is used when the object must be {\em written} in a
machine readable format (as with {\tt write});
\item \Indextt{TK\_MODE} is used when the object must be passed to a Tk
command. This permits to customize the way a Scheme object is converted to a
string when discussing with the Tk library.
\end{itemize}
\paragraph*{Note:} {\tt  display\_fct} can be set to NULL. In this
case the interpreter uses a default printing format. This default
format print the name of the type (found in the {\tt type\_name})
followed by an hexadecimal address.
\end{Lentry}
\subsubsection{Registering the new type}

Once a {\tt STk\_extended\_scheme\_type} structure is defined, the new type can
be registered into the interpreter. Registering a new type is done by the 
\Indextt{STk\_add\_new\_type} function. 
The prototype of this function is given below
\begin{Code}
\begin{listing}[200]{2}
int STk_add_new_type(STk_extended_scheme_type *p)
\end{listing}
\end{Code}
The integer returned by this function is the (unique) key associated to the new
type. This key is stored in each cell of the new type.

We have now enough material to define the {\tt
STk\_extended\_scheme\_type} for the new type {\em stack}. This
declaration can be done in the following way:
\begin{Code}
\begin{listing}[200]{2}
static void mark_stack(SCM p);
static void free_stack(SCM p);
static void display_stack(SCM s, SCM port, int mode);
static int tc_stack;
\end{listing}
\end{Code}
\begin{Code}
\begin{listing}[200]{2}
static STk_extended_scheme_type stack_type = {
  "stack",              /* name */
  0,                    /* is_procp */
  mark_stack,           /* gc_mark_fct */
  free_stack,           /* gc_sweep_fct */
  NULL,                 /* apply_fct */
  display_stack         /* display_fct */
};
\end{listing}
\end{Code}

This definition tells the interpreter that the new type is not a
procedure (field  {\tt is\_procp} is set to 0). Consequently, the 
{\tt apply\_fct} is set to NULL. Note that a display function is
provided here. It permits to used a customized printing
function. 

\subsubsection{New type instances creation}

Creation of a new instance of the extended type necessitates the
definition of a constructor function. This constructor obeys always
the same framework. First you have to create a new cell with the
\Indextt{NEWCELL} macro. This macro has two parameters, a {\tt SCM}
object which will point the new cell and the type of the cell to
create. The second argument is generally equal to the value returned by 
\Indextt{STk\_add\_new\_type}. 
Once the cell is created, we have generally to (dynamically) allocate
a C structure which contains the informations which are necessary to
implement the new type. Dynamic allocation  can be done with the
function \Indextt{STk\_must\_malloc}. The area returned by 
\Indextt{STk\_must\_malloc} must be stored in the {\tt data} field of
the new cell. This field can be accessed with the \Indextt{EXTDATA}
macro. 

Let's go back to the stack example. We can now define a new primitive
function to make a new stack. Provided that the global variable {\tt tc\_stack}
already contains the value returned by {\tt STk\_add\_new\_type}, we
can write

\begin{Code}
\begin{listing}[200]{2}
#define STACKP(x)       (TYPEP(x, tc_stack))
#define NSTACKP(x)      (NTYPEP(x, tc_stack))
#define STACK(x)        ((Stack *) EXTDATA(x))

typedef struct {
  int len;
  SCM values;
} Stack;
\end{listing}
\end{Code}
\begin{Code}
\begin{listing}[200]{2}
static PRIMITIVE make_stack(void)
{
  SCM z;

  NEWCELL(z, tc_stack);
  EXTDATA(z)       = STk_must_malloc(sizeof(Stack));
  STACK(z)->len    = 0;
  STACK(z)->values = NIL;
  return z;
}
\end{listing}
\end{Code}

Here, the {\tt Stack} structure is used to represent a stack. This
structure contains two fields: {\tt len} and {\tt values}. Since the latter
field is a {\tt SCM} object, it must be recursively marked when a
stack is marked. We can now define the utility function necessary for
the GC:
\begin{Code}
\begin{listing}[200]{2}
static void mark_stack(SCM p)
{
  STk_gc_mark(STACK(p)->values);
}

static void free_stack(SCM p)
{
  free(EXTDATA(p));
}
\end{listing}
\end{Code}

To terminate with this example, we give below the code of the primitive 
{\tt stack\_push!}. Other primitive are built in the same fashion and
will not be described here. A complete listing of the stack
implementation is given in appendix.
\begin{Code}
\begin{listing}[200]{2}
static PRIMITIVE stack_push(SCM s, SCM val)
{
  Stack *sp;

  if (NSTACKP(s)) STk_err("stack-push: bad stack", s);

  sp         =  STACK(s);
  sp->len   += 1;
  sp->values = Cons(val, sp->values);
  
  return UNDEFINED;
}
\end{listing}
\end{Code}

\subsection{Definition of a C extended type}

The {\stk} interpreter permits to handle C pointers as first class
objects. 
[for eg: find an example to explain how it works: gdbm?]

\subsection{About memory: Common pitfalls}


\section{Loading an extension}

{\stk} support dynamic loading for several architectures/systems. The
way to provide dynamic loadable modules is different from one system
to another and you will have to adapt what is said here to the
conventions used by your system, architecture or compiler. 
Static loading can be used for systems which doesn' support dynamic
loading (such as Ultrix) or for which the interpreter doesn't support
yet dynamic loading.

\paragraph*{Note:} {\stk) also supports the \Index{DLD Gnu package} for
dynamic loading. DlD is a library package of C functions that performs
"dynamic link editing".  Since the time to load dynamically a module
with this package is rather long, it is preferred to avoid to use it.
However, this package is the only way to provide dynamic loading on
Linux systems which don't support the \Index{ELF} format (versions 1.0
to 1.2). Since the ELF format is becoming the new standard for Linux,
this package will be no more necessary in the future.

The last version of the DLD package can be found at several places:
{\tt
\begin{itemize}
\item ftp-swiss.ai.mit.edu:pub/scm
\item prep.ai.mit.edu:/pub/gnu/jacal
\item ftp.cs.indiana.edu:/pub/scheme-repository/imp/SCM-support
\end{itemize}
}

We suppose here that we want to include the {\tt posix} module defined
in section\ref{simple-example} into the {\stk} interpreter. 
\begin{Lentry}

\item [Dynamic Loading]
If the system running {\stk} supports dynamic loading (and if the
interpreter has been compiled with dynamic loading support), you
compile your source file file to make a {\em \Index{shared object}}
file. On SunOs~4.1, for instance, this can be done by compiling the
module with the {\em \Index{pic compilation option}} (pic stands for 
{\em position independent code}). Once compilation is done, you can
pre-load your file with the line
\begin{Code}
\begin{listing}[200]{2}
ld -assert pure-text -o time.so time.o
\end{listing}
\end{Code}
This will produce a file name {\tt posix.so} which can be loaded with the 
\Indextt{load} Scheme primitive procedure. The {\tt load} primitive recognizes
that this file is a shared object and calls a function whose
name the concatenation of the string 
{\tt STk\_init\_}\index{STk\_init\_ prefix} and the base name of file
loaded. Thus, loading the file {\tt posix.so} impies the call of a
pimitive whose name is {\tt STk\_init\_posix}. 

Look at {\tt Src/Extensions} directory to see some examples of shared
object construction.

\paragraph*{Note:} when the {\stk} is built, the \Indextt{Makefile}  in
the {\tt Src/Extensions} is customized for your system/compiler. A
simple way to determine the options you have to use for compiling your
program consist to run the \Indextt{make} command on one of the file present in
this directory. For instance, issuing the following command
\begin{Code}
\begin{listing}[200]{2}
make -n posix.so
\end{listing}
\end{Code}
on a Linux box using the DLD package will output the following lines:
\begin{Code}
\begin{listing}[200]{2}
gcc  -g -DSTk_CODE -DUSE_DLD -DLINUX   -DHAVE_UNISTD_H=1 \
        -DHAVE_SIGACTION=1 -I../Tk -I../Tcl -I../Src -I../Mp \
        -I/usr/X11R6/include  -c posix.c -o posix.o
ld -r -o posix.so posix.o
\end{listing}
\end{Code}

\item [Static loading]
A C module which define a new type can also be statically loded in the
interepreter. To load your module, you have to modify the {\tt Src/Makefile}
or {\tt  Snow/Makefile}. Once yo have added your extension object in
the \Indextt{USER\_OBJ} variable , you must modify the file
\Indextt{Src/userinit.c} to add you initialization (and eventually
cleanup) code. The call to your initialization function must be done in
the \Indextt{STk\_user\_init} C function. Once this is done, you can
run the {\tt make} command again to build the extended interpreter.
\end{Lentry}

\section{Adding new Tk widgets}

\subsection{Widget compilation}

Adding a new Tk widget to the {\stk} interpreter is generally a simple
{\em hack}. Most of the time, extension widgets written
for Tcl/Tk can be added to the {\stk} interpreter without modifying
the source code of the widget. However, there is no unique method to
add a widget to the Tcl interpreter; consequently, what is given below
is a set of hints to widget integration rather than a {\em always
working recipe}. To illustrate this section, we will see how we can add
the {\em fscale} widget (a floating-point scale widget available on the
Tcl/Tk repository in the {\tt tkFScale-?.?.tar.gz} file)\footnote{
This widget is now integrated in the standard Tk4.0; it is provided as
an extension widget with Tk3.6.}.

Generally, the code of a Tcl/Tk extension widget can be divided in two
parts: the code which implement widget's behavior and the extension
initialization code. Extension initialization code, in Tcl/Tk, must be
placed in the procedure {\tt Tcl\_AppInit} which is located in the file
{\tt tkAppInit.c}. If the extension package adds a lot of widgets, it
generally defines a function to do all the initializations. On the
other hand, if the extension only defines a single widget, the
extension code generally consists to call the C function 
\Indextt{Tcl\_CreateCommand} for each new widget defined in the
extension. {\tt Tcl\_CreateCommand} is the Tcl standard way to add a
new command. This function also exists in the {\stk} interpreter; it
creates a new \Index{Tk command} object~\cite{Gallesio95-1}. The
prototype of this function is:
\begin{Code}
\begin{listing}[200]{2}
void Tcl_CreateCommand(Tcl_Interp *interp,
                       char *cmdName, 
                       Tcl_CmdProc *proc,
                       ClientData clientData,
                       Tcl_CmdDeleteProc *deleteProc));
\end{listing}
\end{Code}

For {\stk},
\begin{itemize}
\item the {\tt interp} is always the global variable
                \Indextt{Stk\_main\_interp}
\item {\tt cmdName} is the name of the widget in the Scheme world.
\item {\tt proc} is the name of the C function which implement the {\em Tk command}
\item {\tt clientData} are informations which are associated to the
widget code. For a new widget, {\tt clientData} can generally set to 
the result of 
\begin{Code}
\begin{listing}[200]{2}
Tk_MainWindow(Stk_main_interp);
\end{listing}
\end{Code}
\item {\tt deleteProc} is a function which is called when the widget
is destroyed. You generally don't need to change the value of this
parameter (which is often set to {\tt NULL}). 
\end{itemize}

The usual way to integrate this initialization code in a Tcl interpreter
consists to patch the {\tt tkAppInit.c} file to add the call to the
initialization (or the {\tt Tcl\_CreateCommand}) function.
To add an extension written for Tcl/Tk to {\stk}, all that is needed
consists to adapt the initialization code for {\stk}. For example, the
{\tt fscale} widget initialization code adds the following call in the
body of the {\tt Tcl\_AppInit} function:
\begin{Code}
\begin{listing}[200]{2}
Tcl_CreateCommand(interp, "fscale", Tk_FScaleCmd, 
                  (ClientData) main, 
                  (void (*)()) NULL);
\end{listing}
\end{Code}

For {\stk}, this call can be written 
\begin{Code}
\begin{listing}[200]{2}
Tcl_CreateCommand(STk_main_interp, "fscale", Tk_FScaleCmd, 
                  (ClientData) Tk_MainWindow(STk_main_interp), 
                  (void (*)()) NULL);
\end{listing}
\end{Code}

This call must executed before trying to create a new {\tt fscale}
widget.

\subsection{Widget linking}

The cleaner way to add a new widget to {\stk} consists to define a
special C module for this widget. Defining the widget in a C module
allows us to make the new widget dynamically loadable. The code for 
making the {\tt fscale} widget dynamically loadable could be:

\begin{Code}
\begin{listing}[200]{2}
/* Contents of the file fscale.c */
#ifndef USE_TK
#define USE_TK
#endif
\end{listing}
\end{Code}
\begin{Code}
\begin{listing}[200]{2}
#include <stk.h>
/* 
 *include the widget source code. Ugly but this avoid to have two
 * source files to link
 */
#include "tkFScale.c"
\end{listing}
\end{Code}
\begin{Code}
\begin{listing}[200]{2}
PRIMITIVE STk_init_fscale(void)
{
  Tcl_CreateCommand(STk_main_interp, 
                    "fscale", 
                    Tk_FScaleCmd, 
                    (ClientData) Tk_MainWindow(STk_main_interp), 
                    (void (*)()) NULL);
}
\end{listing}
\end{Code}

\section{Extending the interpreter with C++}

[for eg : Identify the problems]

\section{Embedding the STk interpreter}

[for eg: This parts need some work in the interpreter]



%%%%%%%%%%%%%%%%%%%%%%%%%%%%%%%%%%%%%%%%%%%%%%%%%%%%%%%%%%%%%%%%%%%%%%%%%%%%%%

\pagebreak
{\Large\bf Appendix}
\par
Hereafter is the complete code for the stack type discussed in \ref{stack}
{\footnotesize
\verbatiminput{stack.c}
}

%%%%%%%%%%%%%%%%%%%%%%%%%%%%%%%%%%%%%%%%%%%%%%%%%%%%%%%%%%%%%%%%%%%%%%%%%%%%%%
\bibliography{bibliography}
%
% A note on how to extend the STk interpreter
%
% Copyright � 1993-1999 Erick Gallesio - I3S-CNRS/ESSI <eg@unice.fr>
%
% Permission to use, copy, modify, distribute,and license this
% software and its documentation for any purpose is hereby granted,
% provided that existing copyright notices are retained in all
% copies and that this notice is included verbatim in any
% distributions.  No written agreement, license, or royalty fee is
% required for any of the authorized uses.
% This software is provided ``AS IS'' without express or implied
% warranty.
%
%           Author: Erick Gallesio [eg@unice.fr]
%    Creation date:  in 1993
% Last file update:  3-Sep-1999 19:36 (eg)
%

\documentclass[10pt]{article}
\usepackage{a4wide}
\usepackage{fancyheadings}
\usepackage{fancybox}
\usepackage{eg-commands}

\pagestyle{fancyplain}
\makeindex\parindent0pt\parskip2mm
\begin{document}
\bibliographystyle{plain}

\title{Extending the ST{\large\bf{K}} interpreter}
\author{Erick Gallesio \\
Universit\'e de Nice~~-~~Sophia-Antipolis \\
Laboratoire I3S - CNRS URA 1376 - ESSI. \\
Route des Colles\\
B.P. 145\\
06903 Sophia-Antipolis Cedex - FRANCE\\[3mm]
email: eg@unice.fr}
\date{July 1995}
\maketitle

\begin{abstract}
This document describes how to extend the {\stk} interpreter with new
primitives procedures and/or new types. Extending the interpreter can be done
by writing new {\em modules} in C. New C code can be statically linked to the
core interpreter or dynamically loaded on operating systems which support
shared libraries. This document also presents how to integrate new
Tk widgets written for the Tcl interpreter in {\stk}.
\end{abstract}


\pagebreak
\tableofcontents
\pagebreak
\section{Introduction}
This document describes how to extend the {\stk}\cite{Gallesio93-1}
interpreter using the C language\cite{Kernighan:CPL88}. To begin, we
will start with a simple extension which will only consist to add some
simple new primitives to the interpreter. Second section will describe
how to add a new type (and the primitives for manipulating this new
type). Another interesting extension consists to add new kind of
primitives (i.e.  primitives which evaluate their argument in
particular way). This kind of extension will be discussed in the third
section.  Fourth section discusses how to add a new widget to the
interpreter. Calling some Scheme code from a C function is showed in
section 5. And last, we will show how to load an extension at load
time. This facility will permit to extend the {\stk} interpreter
without having to recompile it, on systems which support dynamic
loading.

\section{Adding new primitives}

\subsection{A simple example}

\label{simple-example}
One of the simpler extension one can wish to do consists to add new
primitives procedures to the interpreter. To illustrate this, suppose
we want to add two new primitives to the interpreter: {\tt posix-time}
and {\tt posix-ctime}.  The former function correspond to the
POSIX.1\cite{POSIX.1-90} function {\tt time}: it returns the number
of seconds elapsed since 00:00:00 on January 1, 1970, Coordinated
Universal Time (UTC).  The latter is a wrapper around the POSIX.1
function {\tt ctime} which returns a string containing the current
time in an human readable format.

First, we will see how to write the new Scheme primitive {\tt posix-time}.
Implementing a new primitive requires to write a new C function which will do
the work. Here, we write the C function {\tt posix\_time} to implement the
Scheme primitive {\tt posix-time}. The code of this function is given below.
\begin{Code}
\begin{listing}[200]{2}
static PRIMITIVE posix_time(void)
{
  return STk_makeinteger((long) time(NULL));
}
\end{listing}
\end{Code}
This function uses the interpreter \Indextt{STk\_makeinteger} function
which converts a C long integer to a {\stk} integer. Once the {\tt
posix\_time} C function is written, we have to bind this new primitive to
the Scheme symbol {\tt posix-time}. This is achieved by the following C
function call.
\begin{Code}
\begin{listing}[200]{2}
STk_add_new_primitive("posix-time", tc_subr_0, posix_time);
\end{listing}
\end{Code}

\paragraph*{Note:} The C type \Indextt{SCM} is used to describe the objects
manipulated in Scheme. \Indextt{PRIMITIVE} is an alias for this type; it
is preferably used when defining a new primitive.

\Indextt{STk\_add\_new\_primitive} tells the interpreter that the Scheme 
symbol {\tt posix-time} must be bound to the (C written) primitive
{\tt posix\_time}.  The constant \Indextt{tc\_subr\_0} used as the second
argument indicates the arity of this primitive. In this case, the
arity of the primitive is 0.

Let's now have a look at the primitive {\tt posix-ctime}. A first
writing of this primitive could be
\begin{Code}
\begin{listing}[200]{2}
static PRIMITIVE posix_ctime(void)
{
  char *s;
  time_t t = time(NULL);

  s = ctime(&t);
  return STk_makestring(s);
}
\end{listing}
\end{Code}

This functions uses another interpreter routine (\Indextt{STk\_makestring})
which takes as parameter a null terminated string and returns a
Scheme string. 

Binding of the scheme symbol {\tt time-string} to the C function 
{\tt get\_time} is done by the call 
\begin{Code}
\begin{listing}[200]{2}
STk_add_new_primitive("posix-ctime", tc_subr_0, posix_ctime);
\end{listing}
\end{Code}

A complete listing of this code is given in Figure~\ref{posix-1}.
Provided that we have done a shared object of this file, and that its
name is {\tt posix.so}, our two new primitives can be loaded
dynamically by:
\begin{Code}
\begin{listing}[200]{2}
(load "time.so")
\end{listing}
\end{Code}

\paragraph*{Notes:}
\begin{itemize}
\item Suffix can be omitted. Suffixes given in the Scheme variable 
\Indextt{*load-suffixes*} gives the order in which suffixes must be
tried for loading a file.  Default value for this variable is
{\tt ("stk" "stklos" "scm" "so")}.

\item When dynamic loading is used, the interpreter try to call a 
function whose name is equal to the string {\tt "STk\_init\_"} followed by the
name of the file, without suffix. Definitions of new primitives are generally done
in this function. Here, the C function in charge of module initialization must
be called {\tt STk\_init\_posix}.
\end{itemize}

%%%%%%%%%%%%%%%%%%%%%%%%%%%%%%%%%%%%%%%%%%%%%%%%%%%%%%%%%%%%%%%%%%%%%%%%%%%%%%
\begin{figure}
\begin{quote} \footnotesize
\begin{alltt}
#include <sys/types.h>
#include <sys/time.h>
#include <time.h>
#include <stk.h>        {\em /* Declaration of STk objects/primitives */}

static PRIMITIVE posix_time(void)
\{
  return STk_makeinteger((long) time(NULL));
\}

static PRIMITIVE posix_ctime(void)
\{
  char *s;
  time_t t = time(NULL);

  s = ctime(&t);
  return STk_makestring(s);
\}

PRIMITIVE STk_init_posix(void)
\{
  STk_add_new_primitive("posix-time",  tc_subr_0, posix_time);
  STk_add_new_primitive("posix-ctime", tc_subr_0, posix_ctime);
  return UNDEFINED;
\}
\end{alltt}
{\caption{A first version of file {\tt posix.c}}}
\label{posix-1}
\vskip2mm\hrule\vskip3mm
\end{quote}
\end{figure} 

\subsection{Passing arguments to a primitive}

This section shows how to pass arguments to a new primitive written in
C. To illustrate our purpose, we will rewrite the primitive {\tt
posix-ctime} to be conform to POSIX.1 (this function should take an
integer, a count of seconds, and should return corresponding date as a
string). A second writing of previous function could be:

\begin{Code}
\begin{listing}[200]{2}
static PRIMITIVE posix_ctime(SCM seconds)
{
  long sec;

  sec = STk_integer_value_no_overflow(seconds);
  return STk_makestring(ctime((time_t *)&sec));
}
\end{listing}
\end{Code}


This function has one parameter since Scheme primitive arity is one. The
C primitives parameters are always \Indextt{SCM} objects. An object of this type is
a pointer to a \Indextt{struct obj}: the type which permits to represent all the
Scheme objects.  The {\tt SCM} and {\tt struct~obj} types definitions can
be found in the \Indextt{Src/stk.h} header file. 

The first job of this function consists to convert the Scheme
parameter ({\tt seconds}) to a C integer {\tt long int}. This is done with the
function \Indextt{STk\_integer\_value\_no\_overflow}which takes a {\tt SCM}
and returns a {\tt long int}. This functions returns {\tt LONG\_MIN} if
the argument is not a an integer number (or a number which don't fit in the C
representation of a C {\tt long int}). Once this conversion is done,
the rest of the job is similar to the code presented above.

To add this primitive to the global Scheme environment, we have to
change the previous {\tt STk\_add\_new\_primitive} for this primitive by:

\begin{Code}
\begin{listing}[200]{2}
STk_add_new_primitive("posix-ctime", tc_subr_1, posix_ctime);
\end{listing}
\end{Code}

in the init section. This call states that the type of this primitive
is fixed to a \Indextt{tc\_subr\_1} (a arity-1 primitive).

However, this function is not too satisfying, even if close to the
POSIX definition: it obliges to pass a parameter which will be
probably most of the time the result of the primitive {\tt
posix-time} (i.e. the most frequent usage of this function will be

\begin{Code}
\begin{listing}[200]{2}
(posix-ctime (posix-time))
\end{listing}
\end{Code}

which is not very elegant). A better approach consists to allow this
primitive to have a optional parameter. This permits to be at the
same time conform to the POSIX convention and close to Scheme habits. 
The following version implements the {\tt posix-ctime} with a optional
parameter:
\begin{Code}
\begin{listing}[200]{2}
static PRIMITIVE posix_ctime(SCM seconds)
{
  long sec;

  sec = (seconds==UNBOUND) ? time(NULL)
                           : STk_integer_value_no_overflow(seconds);
  return STk_makestring(ctime((time_t *) &sec));
}
\end{listing}
\end{Code}

If the Scheme {\tt posix-ctime} primitive is called with one parameter, it
will be passed to the C function in the {\tt seconds} parameter. If {\tt
posix-ctime} is called without parameter, {\tt seconds} is set to the special
value \Indextt{UNBOUND}. So, the first test in this function consists to set a
correct value to the variable {\tt sec}; this value is either the current
time, either the given integer, depending of the number of parameters
given to {\tt posix-ctime}.

Of course, the type of this new primitive must be changed to allow 0 or 1
parameter. This is done by changing the {\tt tc\_subr\_0} in the previous 
{\tt STk\_add\_new\_primitive} by \Indextt{tc\_subr\_0\_or\_1}.

The following types are available for C primitives:
\begin{itemize}
\item \Indextt{tc\_subr\_0} for arity-0 primitives
\item \Indextt{tc\_subr\_1} for arity-1 primitives
\item \Indextt{tc\_subr\_2} for arity-2 primitives
\item \Indextt{tc\_subr\_3} for arity-3 primitives
\item \Indextt{tc\_subr\_0\_or\_1} for primitives 
      which have 0 or 1 parameter
      (e.g.~{\tt read}). On the C side you have to declare a function
      which takes one {\tt SCM} argument. This argument is set to the
      (evaluated) parameter if present, to {\tt UNBOUND} otherwise.
\item \Indextt{tc\_subr\_1\_or\_2} for 
      primitives which have 1 or 2 parameters
      (e.g.~{\tt write}). Here you have to declare a C function with
      two {\tt SCM} parameters. The first one will contain the first Scheme 
      argument and the second will contain the second argument value
      or {\tt UNBOUND} if omitted.
\item \Indextt{tc\_subr\_2\_or\_3} for 
      primitives which have 2 or 3 parameters
      (there's no primitive of this type in core interpreter).
      Of course, you'll have to declare a C function with three {\tt SCM}
      parameters. Apart that, conventions are the same has before.
\item \Indextt{tc\_lsubr} for primitives which have a variable number of
      arguments. Actuals arguments are collected in a list which is given as the
      first argument of the C primitive. The second argument of the C function
      is an integer counting the actual number of arguments given to the
      primitive. Hence, the signature of the C function which implement a {\tt
      tc\_lsubr} must be
\begin{quote}
\begin{verbatim}
PRIMITIVE function(SCM arglist, int argcount);
\end{verbatim}
\end{quote}
      Note that all the Scheme arguments are evaluated during the
      construction of the list which is passed to the C function.
\item \Indextt{tc\_fsubr} is similar to {\tt tc\_lsubr} except that
      arguments are not evaluated. On the C side, you have to declare a function
      with three {\tt SCM} parameters: the list of (non evaluated) arguments,
      the current environment and the length of the arguments list. The
      signature of the C function which implement a {\tt tc\_fsubr} must be
\begin{quote}
\begin{verbatim}
PRIMITIVE function(SCM arglist, SCM env, int argcount);
\end{verbatim}
\end{quote}
      See \ref{evaluating-args} more details about {\tt tc\_fsubr}.

\item {\tt tc\_tkcommand} for primitives which follow  the Tcl
      command argument passing style (i.e.~{\em \`a la} {\tt argc/argv}). This
      is this kind of procedure that will be used for to add new widgets in the
      {\stk} interpreter. See \ref{new-widget} and \cite{ouster-book} for more
      details.
\end{itemize}

To illustrate how to write a {\tt tc\_lsubr} primitive, let's have a look at the
code, given below, of the function which implement the Scheme primitive {\tt
vector}:
\begin{quote}{\small
\begin{verbatim}
PRIMITIVE STk_vector(SCM arglist, int argcount)
{
  int j;
  SCM z = STk_makevect(argcount, NULL);
  
  for (j = 0; j < argcount; j++, arglist=CDR(arglist)) {
    VECT(z)[j] = CAR(arglist);
  }
  return z;
}
\end{verbatim}
}\end{quote}

This function receives the values passed to the {\tt vector} primitives in the
list arglist (the length of this list is stored in {\tt argcount}). This
function uses \Indextt{STk\_makevect} which returns a Scheme vector. Its first
argument is the length of the vector and its second argument is the initial
value of the vector's elements.  Next section will show how to implement a
primitive which evaluates itself its parameters (i.e. a {\tt tc\_fsubr}
primitive.

\subsection{Evaluating arguments}

In some circumstances it could be useful to add new primitives which don't
evaluate their arguments. This permits to add new control structures to the
interpreter. To illustrate this, we will add two new primitives to the {\stk}
interpreter: {\tt when} and {\tt unless}. As explained in the preceding
section, the C functions which will implement those control structures must be
of type \Indextt{tc\_fsubr}. A {\tt tc\_fsubr} primitive, on the C
side, is given three parameters when called: 
\begin{enumerate}
\item a list of its (non evaluated) parameters, 
\item the local environment when it was called (and in which
evaluations should generally take place),
\item the length of the parameters list.
\end{enumerate}

The C function can step through its parameter list using the C macros
{\tt CAR}, {\tt CDR} and {\tt NULLP} (which do the obvious work) and evaluates elements
of this list as needed. Evaluation of an expression can be done with
the \Indextt{STk\_eval} C function. {\tt STk\_eval} takes two parameters: the
expression to evaluate and the environment in which evaluation takes
place (the \Indextt{NIL} variable, by convention, denotes the Index{global environment}).

\paragraph*{Note:} {\bf a list of arguments is always a proper list. You don't need to test
if it is well formed.}

Hereafter is the code of the {\tt when} primitive.
\begin{Code}
\begin{listing}[200]{2}
static PRIMITIVE when(SCM l, SCM env, int argcount)
{
  SCM res = UNDEFINED;

  if (argcount > 1) {
    if (STk_eval(CAR(l), env) != Ntruth) {
      for (l = CDR(l); !NULLP(l); l = CDR(l)) {
        res = STk_eval(CAR(l), env);
      }
    }
  }
  return res;
}
\end{listing}
\end{Code}

\noindent
Some points to note here:
\begin{itemize}
\item \Indextt{UNDEFINED} is an interpreter {\em constant}. It serves to denote
the notion of {\em ``\Index{unspecified result}''} of {\rrrr}.
\item \Indextt{Truth} and \Indextt{Ntruth} are two global {\em constants} of the
interpreter which denote respectively the \Indextt{\#t} and \Indextt{\#f} Scheme
constants.
\end{itemize}

\noindent
Figure~\ref{when} shows a complete implementation of {\tt when} and {\tt unless}.

\begin{figure}
\begin{quote}\footnotesize
\begin{verbatim}
#include <stk.h>

static PRIMITIVE when(SCM l, SCM env, int argcount)
{
  SCM res = UNDEFINED;

  if (argcount > 1) {
    if (STk_eval(CAR(l), env) != Ntruth) {
      for (l = CDR(l); !NULLP(l); l = CDR(l)) {
        res = STk_eval(CAR(l), env);
      }
    }
  }
  return res;
}

static PRIMITIVE unless(SCM l, SCM env, int argcount)
{
  SCM res = UNDEFINED;

  if (argcount > 1) {
    if (STk_eval(CAR(l), env) == Ntruth) {
      for (l = CDR(l); !NULLP(l); l = CDR(l)) {
        res = STk_eval(CAR(l), env);
      }
    }
  }
  return res;
}

PRIMITIVE STk_init_when_unless(void)
{
  add_new_primitive("when",        tc_fsubr, when);
  add_new_primitive("unless",      tc_fsubr, unless);
  return UNDEFINED;
}
\end{verbatim}
\end{quote}
{\caption{Source listing of file {\tt when\_unless.c}}}
\label{when}
\vskip2mm\hrule\vskip3mm
\end{figure}

\subsection{Signaling errors}
For now, only one function is provided to signal errors: \Indextt{STk\_err}. This
function takes two parameters: a C string which constitutes the body of the
message and a Scheme object (a {\tt SCM} pointer) designating the {\em
erroneous} object.  If the second argument is \Indextt{NIL}, it will not be
printed. Execution of the function {\tt STk\_err} never returns.  It provokes a
jump at the start of the top-level loop. Hereafter, is a new implementation of
the {\tt when} function which uses {\tt STk\_err} when given an erroneous
arguments list.
\begin{Code}
\begin{alltt}
static PRIMITIVE when(SCM l, SCM env, int argcount)
\{
  SCM res = UNDEFINED;

  switch (argcount) \{
    case 0:  STk_err("when: no argument list given", NIL);
    case 1:  STk_err("when: null body", NIL);
    default: {\em /* Argument list is well formed.
              * Evaluate each expression of the body 
              */}
             if (STk_eval(CAR(l), env) != Ntruth)
               for (l = CDR(l); !NULLP(l); l = CDR(l)) 
                 res = STk_eval(CAR(l), env);
  \}
  return res;
\}
\end{alltt}
\end{Code}

\section{Variables}

This section shows how you can access a Scheme variable within C code. It also
shows how you can connect a Scheme and C variable such that modifying it in
Scheme will modify the associated variable and {\em vice versa}.

\subsection{Scheme Symbols and Variables}

\begin{Lentry}
\item[Defining a symbol]
Interning a symbol in the global table of symbols is done with the
\Indextt{STk\_intern} C function. Since this function is often used, you can use
the C macro \Indextt{Intern} as a shortcut. The result of Intern is the {\tt SCM}
object which denotes the scheme symbol associated to the C string passed as parameter.
For example, assigning the list 
\begin{Code}
\begin{listing}[200]{2}
'(green orange red)
\end{listing}
\end{Code}
to the C variable {\tt fire} can be done by
\begin{Code}
\begin{listing}[200]{2}
SCM fire =  Cons(Intern("green"),
                 Cons(Intern("orange"),
                      Cons(Intern("red"), NIL)));
\end{listing}
\end{Code}
Since this notation is difficult to read, some macros have been defined in
\Indextt{Src/stk.h} for building list. These macros are called \Indextt{LISTx} where
x is a number (comprised between 1 and 9) which represent the length of the list
to create. Thus, the previous example could have been written as 
\begin{Code}
\begin{listing}[200]{2}
SCM fire =  LIST3(Intern("green"), Intern("orange"), Intern("red"));
\end{listing}
\end{Code}

\item[Reading a variable]
Reading a variable in Scheme corresponds in fact to look at the value associated
to a symbol. The value associated to a symbol can be obtained with the 
\Indextt{STk\_get\_symbol\_value} C macro. This macro returns a {\tt SCM} object
which correspond to the value associated to the symbol whose name is equal to
the parameter string. \Indextt{STk\_get\_symbol\_value} returns the special
value \Indextt{UNBOUND} is this symbols has no value in the global
environment. The following piece of code
\begin{Code}
\begin{listing}[200]{2}
{
  SCM val = STk_get_symbol_value("foo");

  if (val == UNBOUND)
    STk_err("foo is undefined", NIL);
  else
    STk_display(val, UNBOUND);
}
\end{listing}
\end{Code}
displays the value of the {\tt foo} symbol, or a message is {\tt foo} is undefined in the
global environment. Note the use of the \Indextt{STk\_display} function which
implement the behavior of the Scheme {\tt display} primitive. This call correspond to
a call to {\tt display} with only one parameter, since second parameter is set
to {\tt UNBOUND} (output is done on the standard output port in this case).

\item[Setting a variable]
Setting a Scheme variable corresponds to associate a new value to a symbol. The
value of a symbol can be set with the \Indextt{STk\_set\_symbol\_value} C macro.
For example, 
\begin{Code}
\begin{listing}[200]{2}
STk_set_symbol_value("bar", STk_makeinteger(3L));
\end{listing}
\end{Code}
sets the value of the {\tt bar} symbol to the integer 3. Note that you can set a
symbol in C without using a {\tt define} form as it is necessary in Scheme.
\end{Lentry}

\subsection{Connecting Scheme and C variables}

When building a specialized interpreter, it could be useful to have a variable
you can access both in Scheme an in C. Modifying such a variable in C must
modify the Scheme associated variable and, symmetrically, modifying it in Scheme
must modify the corresponding C variable. One way to do this connection consists
to create a special Scheme variable whose content is read/written by a special
getter/setter. Definition of such a variable, is done by calling the function
\Indextt{STk\_define\_C\_variable}. The C prototype for this function is
\begin{Code}
\begin{listing}[200]{2}
void STk_define_C_variable(char *var, 
                           SCM  (*getter)(char *var), 
                           void (*setter)(char *var, SCM value));
\end{listing}
\end{Code}

The following piece of code shows how we can connect the Scheme variable {\tt *errno*} to the C
variable {\tt errno}:

\begin{Code}
\begin{listing}[200]{2}
static SCM get_errno(char *s)
{
  return STk_makeinteger((long) errno);
}
\end{listing}
\end{Code}
\begin{Code}
\begin{listing}[200]{2}
static void set_errno(char *s, SCM value)
{
  long n = STk_integer_value_no_overflow(value);

  if (n == LONG_MIN) Err("setting *errno*: bad integer", value);
  errno = n;
}
\end{listing}
\end{Code}
\begin{Code}
\begin{listing}[200]{2}
{
  ...
  STk_define_C_variable("*errno*", get_errno, set_errno);
  ...
}
\end{listing}
\end{Code}

After this call to {\tt STk\_define\_C\_variable}, reading ({\em
resp}. writing) the value of the {\tt *errno*} Scheme variable calls the
{\tt get\_errno} ({\em resp}. {\tt set\_errno}) C function.

\section{Calling Scheme from C}

Sometimes, it could be necessary to execute some Scheme code from a C function.
If the Scheme function you have to call is a primitive, it is preferred to call
directly the C function which implement it. To know the name of the C function
which implement a Scheme primitive, you'll have to look in the C file {\tt
primitive\.c} which contains the list of all the primitives of the core
interpreter. If the Scheme code you want to execute is not a call to a
primitive, it is generally easier to put your code in a C string and call the C
function \Indextt{STk\_eval\_C\_string}. This function takes two parameters: the string
to evaluate and the environment in which evaluation must take place. As for {\tt
STk\_eval}, a \Indextt{NIL} value for the environment denotes the global
environment. Suppose, for instance, that you have already written in Scheme the
{\tt fact} procedure; evaluating the factorial of 10 can be done in C with:
\begin{Code}
\begin{listing}[200]{2}
STk_eval_string("(fact 10)", NIL);
\end{listing}
\end{Code}
This call returns a pointer on a Scheme object (a {\tt SCM} pointer) containing
the result of the evaluation. If an error occurs during evaluation. It is
signaled to the user and the constant NULL is returned by {\tt STk\_eval\_string}.

\section{Adding new types}

This sections discusses how to add a new type to the {\stk} interpreter. Interested
reader can find some new types definitions in the {\tt Extensions} directory of
STk. {\stklos}, in particular, is written as an extended type whose definition
is dynamically done as soon as {\em objects} are needed. Hash tables, processes
and sockets are other examples of extended types. 

\subsection{Definition of a Scheme extended type}

Defining a scheme extended type is a little bit more complicated than
defining new primitives since it implies to take into account how this
new type interact with the GC (\Index{Garbage Collector}). Note that
until now we have not discussed about GC problems since the
interpreter is able to hide you it, as far as you don't define new
types. 

To illustrate the discussion, we will show how to add the {\em stack}
type to the {\stk} interpreter in this section. The complete code for
this section can be found in appendix.\label{stack}

\subsubsection{How the GC works}

Before showing how to define a new Scheme type, it is important to understand
how the GC works. First a certain number of cells are created\footnote{by
default 20~000; Use the -cells option of the interpreter to change this
default}. When the interpreter needs a new cell, in the {\tt cons} primitive for
instance, it will take an unused cell in the pool of pre-allocated cells. If no
more cell is available in this area, the GC is called. Its works is divided in
two phases. First phase consists to mark all the cells which are currently in
use. Finding the cells which are in used is done by marking recursively all the
object which are accessible from
\begin{itemize}
    \item the Scheme symbol table,
    \item the registers used by the program,  
    \item the C stack,
    \item global variables of type {\tt SCM}.
\end{itemize}
Marking phase is recursive; that means that if a variable denotes a list, all
the elements of this list have to be marked , to avoid that the GC frees some of
them. Of course, the recursive call for marking the component of a cell depends
on the cell's type. This first phase is called the {\em \Index{marking phase}}.

The second phase of the GC is called the {\em \Index{sweeping phase}}. It is
relatively simple: each allocated cells whose mark bit is unset is placed in the
list of free cells, since nobody points anymore on it.

If no cells can be obtained when the sweeping phase terminates, the pool of
pre-allocated cells will be extended by a new bank of cells. 

\subsubsection{The Extended type data structure}

Defining a new Scheme type consists mainly to define a new
\Indextt{STk\_extended\_scheme\_type} structure and fill in this fields. This
structure is defined as:
\begin{Code}
\begin{listing}[200]{2}
typedef struct {
  char *type_name;
  int  flags;                   
  void (*gc_mark_fct)(SCM x);
  void (*gc_sweep_fct)(SCM x);
  SCM  (*apply_fct)(SCM x, SCM args, SCM env);
  void (*display_fct)(SCM x, SCM port, int mode);
} STk_extended_scheme_type;
\end{listing}
\end{Code}
Each field of this structure is defined below
\begin{Lentry}
\item[type\_name] is a string. It denotes the external name of the new type. The
purpose of this field is mainly for debugging.

\item[flags] is the union of binary constants. For now, only two constants are
defined:
    \begin{itemize}
    \item \Indextt{EXT\_ISPROC} must be set if the new type is a procedure (i.e. if the Scheme
    procedure must answer \#t when called with this object).
    \item \Indextt{EXT\_EVALPARAM} must be set if the new type must evaluates its parameters
    when used as a function.
    \end{itemize}

\item [gc\_mark\_fct] is a pointer to the function which marks objects of the
extended type. The code associated to this function is simple. It consists to
call \index{STk\_gc\_mark} on each field whose type is {\tt SCM} in the type
associated data. This function is automatically called by the interpreter when
it scans all the used cells in the GC \Index{marking phase}.
One example of {\tt gc\_mark\_fct} is given below. 

\item [gc\_sweep\_fct] is a pointer to the function which frees the resources
allocated for representing the new type of object. This function is
automatically called by the GC in the sweeping phase for each cell which is
unused. 

\item [apply\_function] is a pointer to a function which is called when applying
this object to a list of arguments.  This function can only be called if the bit
\Indextt{EXT\_ISPROC} is set. The arguments given to this function are 
evaluated if the \Indextt{EXT\_EVALPARAM} bit is set. Finally, the environment in
which the call is done is passed as the third argument of the {\tt
apply\_function}. It serves principally when the {\tt EXT\_EVALPARAM} bit is
unset.

Set the {\tt apply\_function} to NULL to use the interpreter default apply
function. The default function raises an error when called. You can use the
default function when the new type you define is not a function.

\item [display\_fct] is a pointer to a C function which displays objects of the
new type. The display function has three parameters. The first parameter is the
object to print. The second parameter is the port to which the object must be
printed. Printing an object must be done with one of the following functions
{\tt
\begin{itemize}
\item int  \Index{STk\_getc}(FILE *f);
\item int  \Index{STk\_ungetc}(int c, FILE *f);
\item int  \Index{STk\_putc}(int c, FILE *f);
\item int  \Index{STk\_puts}(char *s, FILE *f);
\item int  \Index{STk\_eof}(FILE *f);
\end{itemize}
}
Those functions are extensions of their C equivalent: they are able to handle the
{\stk} string ports.

The third parameter of the {\tt display\_fct} is a mode constant which can take three
different values. 
\begin{itemize}
\item \Indextt{DSP\_MODE} is used when the object must be {\em displayed} in a
human readable format (as with {\tt display});
\item \Indextt{WRT\_MODE} is used when the object must be {\em written} in a
machine readable format (as with {\tt write});
\item \Indextt{TK\_MODE} is used when the object must be passed to a Tk
command. This permits to customize the way a Scheme object is converted to a
string when discussing with the Tk library.
\end{itemize}
\paragraph*{Note:} {\tt  display\_fct} can be set to NULL. In this
case the interpreter uses a default printing format. This default
format print the name of the type (found in the {\tt type\_name})
followed by an hexadecimal address.
\end{Lentry}
\subsubsection{Registering the new type}

Once a {\tt STk\_extended\_scheme\_type} structure is defined, the new type can
be registered into the interpreter. Registering a new type is done by the 
\Indextt{STk\_add\_new\_type} function. 
The prototype of this function is given below
\begin{Code}
\begin{listing}[200]{2}
int STk_add_new_type(STk_extended_scheme_type *p)
\end{listing}
\end{Code}
The integer returned by this function is the (unique) key associated to the new
type. This key is stored in each cell of the new type.

We have now enough material to define the {\tt
STk\_extended\_scheme\_type} for the new type {\em stack}. This
declaration can be done in the following way:
\begin{Code}
\begin{listing}[200]{2}
static void mark_stack(SCM p);
static void free_stack(SCM p);
static void display_stack(SCM s, SCM port, int mode);
static int tc_stack;
\end{listing}
\end{Code}
\begin{Code}
\begin{listing}[200]{2}
static STk_extended_scheme_type stack_type = {
  "stack",              /* name */
  0,                    /* is_procp */
  mark_stack,           /* gc_mark_fct */
  free_stack,           /* gc_sweep_fct */
  NULL,                 /* apply_fct */
  display_stack         /* display_fct */
};
\end{listing}
\end{Code}

This definition tells the interpreter that the new type is not a
procedure (field  {\tt is\_procp} is set to 0). Consequently, the 
{\tt apply\_fct} is set to NULL. Note that a display function is
provided here. It permits to used a customized printing
function. 

\subsubsection{New type instances creation}

Creation of a new instance of the extended type necessitates the
definition of a constructor function. This constructor obeys always
the same framework. First you have to create a new cell with the
\Indextt{NEWCELL} macro. This macro has two parameters, a {\tt SCM}
object which will point the new cell and the type of the cell to
create. The second argument is generally equal to the value returned by 
\Indextt{STk\_add\_new\_type}. 
Once the cell is created, we have generally to (dynamically) allocate
a C structure which contains the informations which are necessary to
implement the new type. Dynamic allocation  can be done with the
function \Indextt{STk\_must\_malloc}. The area returned by 
\Indextt{STk\_must\_malloc} must be stored in the {\tt data} field of
the new cell. This field can be accessed with the \Indextt{EXTDATA}
macro. 

Let's go back to the stack example. We can now define a new primitive
function to make a new stack. Provided that the global variable {\tt tc\_stack}
already contains the value returned by {\tt STk\_add\_new\_type}, we
can write

\begin{Code}
\begin{listing}[200]{2}
#define STACKP(x)       (TYPEP(x, tc_stack))
#define NSTACKP(x)      (NTYPEP(x, tc_stack))
#define STACK(x)        ((Stack *) EXTDATA(x))

typedef struct {
  int len;
  SCM values;
} Stack;
\end{listing}
\end{Code}
\begin{Code}
\begin{listing}[200]{2}
static PRIMITIVE make_stack(void)
{
  SCM z;

  NEWCELL(z, tc_stack);
  EXTDATA(z)       = STk_must_malloc(sizeof(Stack));
  STACK(z)->len    = 0;
  STACK(z)->values = NIL;
  return z;
}
\end{listing}
\end{Code}

Here, the {\tt Stack} structure is used to represent a stack. This
structure contains two fields: {\tt len} and {\tt values}. Since the latter
field is a {\tt SCM} object, it must be recursively marked when a
stack is marked. We can now define the utility function necessary for
the GC:
\begin{Code}
\begin{listing}[200]{2}
static void mark_stack(SCM p)
{
  STk_gc_mark(STACK(p)->values);
}

static void free_stack(SCM p)
{
  free(EXTDATA(p));
}
\end{listing}
\end{Code}

To terminate with this example, we give below the code of the primitive 
{\tt stack\_push!}. Other primitive are built in the same fashion and
will not be described here. A complete listing of the stack
implementation is given in appendix.
\begin{Code}
\begin{listing}[200]{2}
static PRIMITIVE stack_push(SCM s, SCM val)
{
  Stack *sp;

  if (NSTACKP(s)) STk_err("stack-push: bad stack", s);

  sp         =  STACK(s);
  sp->len   += 1;
  sp->values = Cons(val, sp->values);
  
  return UNDEFINED;
}
\end{listing}
\end{Code}

\subsection{Definition of a C extended type}

The {\stk} interpreter permits to handle C pointers as first class
objects. 
[for eg: find an example to explain how it works: gdbm?]

\subsection{About memory: Common pitfalls}


\section{Loading an extension}

{\stk} support dynamic loading for several architectures/systems. The
way to provide dynamic loadable modules is different from one system
to another and you will have to adapt what is said here to the
conventions used by your system, architecture or compiler. 
Static loading can be used for systems which doesn' support dynamic
loading (such as Ultrix) or for which the interpreter doesn't support
yet dynamic loading.

\paragraph*{Note:} {\stk) also supports the \Index{DLD Gnu package} for
dynamic loading. DlD is a library package of C functions that performs
"dynamic link editing".  Since the time to load dynamically a module
with this package is rather long, it is preferred to avoid to use it.
However, this package is the only way to provide dynamic loading on
Linux systems which don't support the \Index{ELF} format (versions 1.0
to 1.2). Since the ELF format is becoming the new standard for Linux,
this package will be no more necessary in the future.

The last version of the DLD package can be found at several places:
{\tt
\begin{itemize}
\item ftp-swiss.ai.mit.edu:pub/scm
\item prep.ai.mit.edu:/pub/gnu/jacal
\item ftp.cs.indiana.edu:/pub/scheme-repository/imp/SCM-support
\end{itemize}
}

We suppose here that we want to include the {\tt posix} module defined
in section\ref{simple-example} into the {\stk} interpreter. 
\begin{Lentry}

\item [Dynamic Loading]
If the system running {\stk} supports dynamic loading (and if the
interpreter has been compiled with dynamic loading support), you
compile your source file file to make a {\em \Index{shared object}}
file. On SunOs~4.1, for instance, this can be done by compiling the
module with the {\em \Index{pic compilation option}} (pic stands for 
{\em position independent code}). Once compilation is done, you can
pre-load your file with the line
\begin{Code}
\begin{listing}[200]{2}
ld -assert pure-text -o time.so time.o
\end{listing}
\end{Code}
This will produce a file name {\tt posix.so} which can be loaded with the 
\Indextt{load} Scheme primitive procedure. The {\tt load} primitive recognizes
that this file is a shared object and calls a function whose
name the concatenation of the string 
{\tt STk\_init\_}\index{STk\_init\_ prefix} and the base name of file
loaded. Thus, loading the file {\tt posix.so} impies the call of a
pimitive whose name is {\tt STk\_init\_posix}. 

Look at {\tt Src/Extensions} directory to see some examples of shared
object construction.

\paragraph*{Note:} when the {\stk} is built, the \Indextt{Makefile}  in
the {\tt Src/Extensions} is customized for your system/compiler. A
simple way to determine the options you have to use for compiling your
program consist to run the \Indextt{make} command on one of the file present in
this directory. For instance, issuing the following command
\begin{Code}
\begin{listing}[200]{2}
make -n posix.so
\end{listing}
\end{Code}
on a Linux box using the DLD package will output the following lines:
\begin{Code}
\begin{listing}[200]{2}
gcc  -g -DSTk_CODE -DUSE_DLD -DLINUX   -DHAVE_UNISTD_H=1 \
        -DHAVE_SIGACTION=1 -I../Tk -I../Tcl -I../Src -I../Mp \
        -I/usr/X11R6/include  -c posix.c -o posix.o
ld -r -o posix.so posix.o
\end{listing}
\end{Code}

\item [Static loading]
A C module which define a new type can also be statically loded in the
interepreter. To load your module, you have to modify the {\tt Src/Makefile}
or {\tt  Snow/Makefile}. Once yo have added your extension object in
the \Indextt{USER\_OBJ} variable , you must modify the file
\Indextt{Src/userinit.c} to add you initialization (and eventually
cleanup) code. The call to your initialization function must be done in
the \Indextt{STk\_user\_init} C function. Once this is done, you can
run the {\tt make} command again to build the extended interpreter.
\end{Lentry}

\section{Adding new Tk widgets}

\subsection{Widget compilation}

Adding a new Tk widget to the {\stk} interpreter is generally a simple
{\em hack}. Most of the time, extension widgets written
for Tcl/Tk can be added to the {\stk} interpreter without modifying
the source code of the widget. However, there is no unique method to
add a widget to the Tcl interpreter; consequently, what is given below
is a set of hints to widget integration rather than a {\em always
working recipe}. To illustrate this section, we will see how we can add
the {\em fscale} widget (a floating-point scale widget available on the
Tcl/Tk repository in the {\tt tkFScale-?.?.tar.gz} file)\footnote{
This widget is now integrated in the standard Tk4.0; it is provided as
an extension widget with Tk3.6.}.

Generally, the code of a Tcl/Tk extension widget can be divided in two
parts: the code which implement widget's behavior and the extension
initialization code. Extension initialization code, in Tcl/Tk, must be
placed in the procedure {\tt Tcl\_AppInit} which is located in the file
{\tt tkAppInit.c}. If the extension package adds a lot of widgets, it
generally defines a function to do all the initializations. On the
other hand, if the extension only defines a single widget, the
extension code generally consists to call the C function 
\Indextt{Tcl\_CreateCommand} for each new widget defined in the
extension. {\tt Tcl\_CreateCommand} is the Tcl standard way to add a
new command. This function also exists in the {\stk} interpreter; it
creates a new \Index{Tk command} object~\cite{Gallesio95-1}. The
prototype of this function is:
\begin{Code}
\begin{listing}[200]{2}
void Tcl_CreateCommand(Tcl_Interp *interp,
                       char *cmdName, 
                       Tcl_CmdProc *proc,
                       ClientData clientData,
                       Tcl_CmdDeleteProc *deleteProc));
\end{listing}
\end{Code}

For {\stk},
\begin{itemize}
\item the {\tt interp} is always the global variable
                \Indextt{Stk\_main\_interp}
\item {\tt cmdName} is the name of the widget in the Scheme world.
\item {\tt proc} is the name of the C function which implement the {\em Tk command}
\item {\tt clientData} are informations which are associated to the
widget code. For a new widget, {\tt clientData} can generally set to 
the result of 
\begin{Code}
\begin{listing}[200]{2}
Tk_MainWindow(Stk_main_interp);
\end{listing}
\end{Code}
\item {\tt deleteProc} is a function which is called when the widget
is destroyed. You generally don't need to change the value of this
parameter (which is often set to {\tt NULL}). 
\end{itemize}

The usual way to integrate this initialization code in a Tcl interpreter
consists to patch the {\tt tkAppInit.c} file to add the call to the
initialization (or the {\tt Tcl\_CreateCommand}) function.
To add an extension written for Tcl/Tk to {\stk}, all that is needed
consists to adapt the initialization code for {\stk}. For example, the
{\tt fscale} widget initialization code adds the following call in the
body of the {\tt Tcl\_AppInit} function:
\begin{Code}
\begin{listing}[200]{2}
Tcl_CreateCommand(interp, "fscale", Tk_FScaleCmd, 
                  (ClientData) main, 
                  (void (*)()) NULL);
\end{listing}
\end{Code}

For {\stk}, this call can be written 
\begin{Code}
\begin{listing}[200]{2}
Tcl_CreateCommand(STk_main_interp, "fscale", Tk_FScaleCmd, 
                  (ClientData) Tk_MainWindow(STk_main_interp), 
                  (void (*)()) NULL);
\end{listing}
\end{Code}

This call must executed before trying to create a new {\tt fscale}
widget.

\subsection{Widget linking}

The cleaner way to add a new widget to {\stk} consists to define a
special C module for this widget. Defining the widget in a C module
allows us to make the new widget dynamically loadable. The code for 
making the {\tt fscale} widget dynamically loadable could be:

\begin{Code}
\begin{listing}[200]{2}
/* Contents of the file fscale.c */
#ifndef USE_TK
#define USE_TK
#endif
\end{listing}
\end{Code}
\begin{Code}
\begin{listing}[200]{2}
#include <stk.h>
/* 
 *include the widget source code. Ugly but this avoid to have two
 * source files to link
 */
#include "tkFScale.c"
\end{listing}
\end{Code}
\begin{Code}
\begin{listing}[200]{2}
PRIMITIVE STk_init_fscale(void)
{
  Tcl_CreateCommand(STk_main_interp, 
                    "fscale", 
                    Tk_FScaleCmd, 
                    (ClientData) Tk_MainWindow(STk_main_interp), 
                    (void (*)()) NULL);
}
\end{listing}
\end{Code}

\section{Extending the interpreter with C++}

[for eg : Identify the problems]

\section{Embedding the STk interpreter}

[for eg: This parts need some work in the interpreter]



%%%%%%%%%%%%%%%%%%%%%%%%%%%%%%%%%%%%%%%%%%%%%%%%%%%%%%%%%%%%%%%%%%%%%%%%%%%%%%

\pagebreak
{\Large\bf Appendix}
\par
Hereafter is the complete code for the stack type discussed in \ref{stack}
{\footnotesize
\verbatiminput{stack.c}
}

%%%%%%%%%%%%%%%%%%%%%%%%%%%%%%%%%%%%%%%%%%%%%%%%%%%%%%%%%%%%%%%%%%%%%%%%%%%%%%
\bibliography{bibliography}
%
% A note on how to extend the STk interpreter
%
% Copyright � 1993-1999 Erick Gallesio - I3S-CNRS/ESSI <eg@unice.fr>
%
% Permission to use, copy, modify, distribute,and license this
% software and its documentation for any purpose is hereby granted,
% provided that existing copyright notices are retained in all
% copies and that this notice is included verbatim in any
% distributions.  No written agreement, license, or royalty fee is
% required for any of the authorized uses.
% This software is provided ``AS IS'' without express or implied
% warranty.
%
%           Author: Erick Gallesio [eg@unice.fr]
%    Creation date:  in 1993
% Last file update:  3-Sep-1999 19:36 (eg)
%

\documentclass[10pt]{article}
\usepackage{a4wide}
\usepackage{fancyheadings}
\usepackage{fancybox}
\usepackage{eg-commands}

\pagestyle{fancyplain}
\makeindex\parindent0pt\parskip2mm
\begin{document}
\bibliographystyle{plain}

\title{Extending the ST{\large\bf{K}} interpreter}
\author{Erick Gallesio \\
Universit\'e de Nice~~-~~Sophia-Antipolis \\
Laboratoire I3S - CNRS URA 1376 - ESSI. \\
Route des Colles\\
B.P. 145\\
06903 Sophia-Antipolis Cedex - FRANCE\\[3mm]
email: eg@unice.fr}
\date{July 1995}
\maketitle

\begin{abstract}
This document describes how to extend the {\stk} interpreter with new
primitives procedures and/or new types. Extending the interpreter can be done
by writing new {\em modules} in C. New C code can be statically linked to the
core interpreter or dynamically loaded on operating systems which support
shared libraries. This document also presents how to integrate new
Tk widgets written for the Tcl interpreter in {\stk}.
\end{abstract}


\pagebreak
\tableofcontents
\pagebreak
\section{Introduction}
This document describes how to extend the {\stk}\cite{Gallesio93-1}
interpreter using the C language\cite{Kernighan:CPL88}. To begin, we
will start with a simple extension which will only consist to add some
simple new primitives to the interpreter. Second section will describe
how to add a new type (and the primitives for manipulating this new
type). Another interesting extension consists to add new kind of
primitives (i.e.  primitives which evaluate their argument in
particular way). This kind of extension will be discussed in the third
section.  Fourth section discusses how to add a new widget to the
interpreter. Calling some Scheme code from a C function is showed in
section 5. And last, we will show how to load an extension at load
time. This facility will permit to extend the {\stk} interpreter
without having to recompile it, on systems which support dynamic
loading.

\section{Adding new primitives}

\subsection{A simple example}

\label{simple-example}
One of the simpler extension one can wish to do consists to add new
primitives procedures to the interpreter. To illustrate this, suppose
we want to add two new primitives to the interpreter: {\tt posix-time}
and {\tt posix-ctime}.  The former function correspond to the
POSIX.1\cite{POSIX.1-90} function {\tt time}: it returns the number
of seconds elapsed since 00:00:00 on January 1, 1970, Coordinated
Universal Time (UTC).  The latter is a wrapper around the POSIX.1
function {\tt ctime} which returns a string containing the current
time in an human readable format.

First, we will see how to write the new Scheme primitive {\tt posix-time}.
Implementing a new primitive requires to write a new C function which will do
the work. Here, we write the C function {\tt posix\_time} to implement the
Scheme primitive {\tt posix-time}. The code of this function is given below.
\begin{Code}
\begin{listing}[200]{2}
static PRIMITIVE posix_time(void)
{
  return STk_makeinteger((long) time(NULL));
}
\end{listing}
\end{Code}
This function uses the interpreter \Indextt{STk\_makeinteger} function
which converts a C long integer to a {\stk} integer. Once the {\tt
posix\_time} C function is written, we have to bind this new primitive to
the Scheme symbol {\tt posix-time}. This is achieved by the following C
function call.
\begin{Code}
\begin{listing}[200]{2}
STk_add_new_primitive("posix-time", tc_subr_0, posix_time);
\end{listing}
\end{Code}

\paragraph*{Note:} The C type \Indextt{SCM} is used to describe the objects
manipulated in Scheme. \Indextt{PRIMITIVE} is an alias for this type; it
is preferably used when defining a new primitive.

\Indextt{STk\_add\_new\_primitive} tells the interpreter that the Scheme 
symbol {\tt posix-time} must be bound to the (C written) primitive
{\tt posix\_time}.  The constant \Indextt{tc\_subr\_0} used as the second
argument indicates the arity of this primitive. In this case, the
arity of the primitive is 0.

Let's now have a look at the primitive {\tt posix-ctime}. A first
writing of this primitive could be
\begin{Code}
\begin{listing}[200]{2}
static PRIMITIVE posix_ctime(void)
{
  char *s;
  time_t t = time(NULL);

  s = ctime(&t);
  return STk_makestring(s);
}
\end{listing}
\end{Code}

This functions uses another interpreter routine (\Indextt{STk\_makestring})
which takes as parameter a null terminated string and returns a
Scheme string. 

Binding of the scheme symbol {\tt time-string} to the C function 
{\tt get\_time} is done by the call 
\begin{Code}
\begin{listing}[200]{2}
STk_add_new_primitive("posix-ctime", tc_subr_0, posix_ctime);
\end{listing}
\end{Code}

A complete listing of this code is given in Figure~\ref{posix-1}.
Provided that we have done a shared object of this file, and that its
name is {\tt posix.so}, our two new primitives can be loaded
dynamically by:
\begin{Code}
\begin{listing}[200]{2}
(load "time.so")
\end{listing}
\end{Code}

\paragraph*{Notes:}
\begin{itemize}
\item Suffix can be omitted. Suffixes given in the Scheme variable 
\Indextt{*load-suffixes*} gives the order in which suffixes must be
tried for loading a file.  Default value for this variable is
{\tt ("stk" "stklos" "scm" "so")}.

\item When dynamic loading is used, the interpreter try to call a 
function whose name is equal to the string {\tt "STk\_init\_"} followed by the
name of the file, without suffix. Definitions of new primitives are generally done
in this function. Here, the C function in charge of module initialization must
be called {\tt STk\_init\_posix}.
\end{itemize}

%%%%%%%%%%%%%%%%%%%%%%%%%%%%%%%%%%%%%%%%%%%%%%%%%%%%%%%%%%%%%%%%%%%%%%%%%%%%%%
\begin{figure}
\begin{quote} \footnotesize
\begin{alltt}
#include <sys/types.h>
#include <sys/time.h>
#include <time.h>
#include <stk.h>        {\em /* Declaration of STk objects/primitives */}

static PRIMITIVE posix_time(void)
\{
  return STk_makeinteger((long) time(NULL));
\}

static PRIMITIVE posix_ctime(void)
\{
  char *s;
  time_t t = time(NULL);

  s = ctime(&t);
  return STk_makestring(s);
\}

PRIMITIVE STk_init_posix(void)
\{
  STk_add_new_primitive("posix-time",  tc_subr_0, posix_time);
  STk_add_new_primitive("posix-ctime", tc_subr_0, posix_ctime);
  return UNDEFINED;
\}
\end{alltt}
{\caption{A first version of file {\tt posix.c}}}
\label{posix-1}
\vskip2mm\hrule\vskip3mm
\end{quote}
\end{figure} 

\subsection{Passing arguments to a primitive}

This section shows how to pass arguments to a new primitive written in
C. To illustrate our purpose, we will rewrite the primitive {\tt
posix-ctime} to be conform to POSIX.1 (this function should take an
integer, a count of seconds, and should return corresponding date as a
string). A second writing of previous function could be:

\begin{Code}
\begin{listing}[200]{2}
static PRIMITIVE posix_ctime(SCM seconds)
{
  long sec;

  sec = STk_integer_value_no_overflow(seconds);
  return STk_makestring(ctime((time_t *)&sec));
}
\end{listing}
\end{Code}


This function has one parameter since Scheme primitive arity is one. The
C primitives parameters are always \Indextt{SCM} objects. An object of this type is
a pointer to a \Indextt{struct obj}: the type which permits to represent all the
Scheme objects.  The {\tt SCM} and {\tt struct~obj} types definitions can
be found in the \Indextt{Src/stk.h} header file. 

The first job of this function consists to convert the Scheme
parameter ({\tt seconds}) to a C integer {\tt long int}. This is done with the
function \Indextt{STk\_integer\_value\_no\_overflow}which takes a {\tt SCM}
and returns a {\tt long int}. This functions returns {\tt LONG\_MIN} if
the argument is not a an integer number (or a number which don't fit in the C
representation of a C {\tt long int}). Once this conversion is done,
the rest of the job is similar to the code presented above.

To add this primitive to the global Scheme environment, we have to
change the previous {\tt STk\_add\_new\_primitive} for this primitive by:

\begin{Code}
\begin{listing}[200]{2}
STk_add_new_primitive("posix-ctime", tc_subr_1, posix_ctime);
\end{listing}
\end{Code}

in the init section. This call states that the type of this primitive
is fixed to a \Indextt{tc\_subr\_1} (a arity-1 primitive).

However, this function is not too satisfying, even if close to the
POSIX definition: it obliges to pass a parameter which will be
probably most of the time the result of the primitive {\tt
posix-time} (i.e. the most frequent usage of this function will be

\begin{Code}
\begin{listing}[200]{2}
(posix-ctime (posix-time))
\end{listing}
\end{Code}

which is not very elegant). A better approach consists to allow this
primitive to have a optional parameter. This permits to be at the
same time conform to the POSIX convention and close to Scheme habits. 
The following version implements the {\tt posix-ctime} with a optional
parameter:
\begin{Code}
\begin{listing}[200]{2}
static PRIMITIVE posix_ctime(SCM seconds)
{
  long sec;

  sec = (seconds==UNBOUND) ? time(NULL)
                           : STk_integer_value_no_overflow(seconds);
  return STk_makestring(ctime((time_t *) &sec));
}
\end{listing}
\end{Code}

If the Scheme {\tt posix-ctime} primitive is called with one parameter, it
will be passed to the C function in the {\tt seconds} parameter. If {\tt
posix-ctime} is called without parameter, {\tt seconds} is set to the special
value \Indextt{UNBOUND}. So, the first test in this function consists to set a
correct value to the variable {\tt sec}; this value is either the current
time, either the given integer, depending of the number of parameters
given to {\tt posix-ctime}.

Of course, the type of this new primitive must be changed to allow 0 or 1
parameter. This is done by changing the {\tt tc\_subr\_0} in the previous 
{\tt STk\_add\_new\_primitive} by \Indextt{tc\_subr\_0\_or\_1}.

The following types are available for C primitives:
\begin{itemize}
\item \Indextt{tc\_subr\_0} for arity-0 primitives
\item \Indextt{tc\_subr\_1} for arity-1 primitives
\item \Indextt{tc\_subr\_2} for arity-2 primitives
\item \Indextt{tc\_subr\_3} for arity-3 primitives
\item \Indextt{tc\_subr\_0\_or\_1} for primitives 
      which have 0 or 1 parameter
      (e.g.~{\tt read}). On the C side you have to declare a function
      which takes one {\tt SCM} argument. This argument is set to the
      (evaluated) parameter if present, to {\tt UNBOUND} otherwise.
\item \Indextt{tc\_subr\_1\_or\_2} for 
      primitives which have 1 or 2 parameters
      (e.g.~{\tt write}). Here you have to declare a C function with
      two {\tt SCM} parameters. The first one will contain the first Scheme 
      argument and the second will contain the second argument value
      or {\tt UNBOUND} if omitted.
\item \Indextt{tc\_subr\_2\_or\_3} for 
      primitives which have 2 or 3 parameters
      (there's no primitive of this type in core interpreter).
      Of course, you'll have to declare a C function with three {\tt SCM}
      parameters. Apart that, conventions are the same has before.
\item \Indextt{tc\_lsubr} for primitives which have a variable number of
      arguments. Actuals arguments are collected in a list which is given as the
      first argument of the C primitive. The second argument of the C function
      is an integer counting the actual number of arguments given to the
      primitive. Hence, the signature of the C function which implement a {\tt
      tc\_lsubr} must be
\begin{quote}
\begin{verbatim}
PRIMITIVE function(SCM arglist, int argcount);
\end{verbatim}
\end{quote}
      Note that all the Scheme arguments are evaluated during the
      construction of the list which is passed to the C function.
\item \Indextt{tc\_fsubr} is similar to {\tt tc\_lsubr} except that
      arguments are not evaluated. On the C side, you have to declare a function
      with three {\tt SCM} parameters: the list of (non evaluated) arguments,
      the current environment and the length of the arguments list. The
      signature of the C function which implement a {\tt tc\_fsubr} must be
\begin{quote}
\begin{verbatim}
PRIMITIVE function(SCM arglist, SCM env, int argcount);
\end{verbatim}
\end{quote}
      See \ref{evaluating-args} more details about {\tt tc\_fsubr}.

\item {\tt tc\_tkcommand} for primitives which follow  the Tcl
      command argument passing style (i.e.~{\em \`a la} {\tt argc/argv}). This
      is this kind of procedure that will be used for to add new widgets in the
      {\stk} interpreter. See \ref{new-widget} and \cite{ouster-book} for more
      details.
\end{itemize}

To illustrate how to write a {\tt tc\_lsubr} primitive, let's have a look at the
code, given below, of the function which implement the Scheme primitive {\tt
vector}:
\begin{quote}{\small
\begin{verbatim}
PRIMITIVE STk_vector(SCM arglist, int argcount)
{
  int j;
  SCM z = STk_makevect(argcount, NULL);
  
  for (j = 0; j < argcount; j++, arglist=CDR(arglist)) {
    VECT(z)[j] = CAR(arglist);
  }
  return z;
}
\end{verbatim}
}\end{quote}

This function receives the values passed to the {\tt vector} primitives in the
list arglist (the length of this list is stored in {\tt argcount}). This
function uses \Indextt{STk\_makevect} which returns a Scheme vector. Its first
argument is the length of the vector and its second argument is the initial
value of the vector's elements.  Next section will show how to implement a
primitive which evaluates itself its parameters (i.e. a {\tt tc\_fsubr}
primitive.

\subsection{Evaluating arguments}

In some circumstances it could be useful to add new primitives which don't
evaluate their arguments. This permits to add new control structures to the
interpreter. To illustrate this, we will add two new primitives to the {\stk}
interpreter: {\tt when} and {\tt unless}. As explained in the preceding
section, the C functions which will implement those control structures must be
of type \Indextt{tc\_fsubr}. A {\tt tc\_fsubr} primitive, on the C
side, is given three parameters when called: 
\begin{enumerate}
\item a list of its (non evaluated) parameters, 
\item the local environment when it was called (and in which
evaluations should generally take place),
\item the length of the parameters list.
\end{enumerate}

The C function can step through its parameter list using the C macros
{\tt CAR}, {\tt CDR} and {\tt NULLP} (which do the obvious work) and evaluates elements
of this list as needed. Evaluation of an expression can be done with
the \Indextt{STk\_eval} C function. {\tt STk\_eval} takes two parameters: the
expression to evaluate and the environment in which evaluation takes
place (the \Indextt{NIL} variable, by convention, denotes the Index{global environment}).

\paragraph*{Note:} {\bf a list of arguments is always a proper list. You don't need to test
if it is well formed.}

Hereafter is the code of the {\tt when} primitive.
\begin{Code}
\begin{listing}[200]{2}
static PRIMITIVE when(SCM l, SCM env, int argcount)
{
  SCM res = UNDEFINED;

  if (argcount > 1) {
    if (STk_eval(CAR(l), env) != Ntruth) {
      for (l = CDR(l); !NULLP(l); l = CDR(l)) {
        res = STk_eval(CAR(l), env);
      }
    }
  }
  return res;
}
\end{listing}
\end{Code}

\noindent
Some points to note here:
\begin{itemize}
\item \Indextt{UNDEFINED} is an interpreter {\em constant}. It serves to denote
the notion of {\em ``\Index{unspecified result}''} of {\rrrr}.
\item \Indextt{Truth} and \Indextt{Ntruth} are two global {\em constants} of the
interpreter which denote respectively the \Indextt{\#t} and \Indextt{\#f} Scheme
constants.
\end{itemize}

\noindent
Figure~\ref{when} shows a complete implementation of {\tt when} and {\tt unless}.

\begin{figure}
\begin{quote}\footnotesize
\begin{verbatim}
#include <stk.h>

static PRIMITIVE when(SCM l, SCM env, int argcount)
{
  SCM res = UNDEFINED;

  if (argcount > 1) {
    if (STk_eval(CAR(l), env) != Ntruth) {
      for (l = CDR(l); !NULLP(l); l = CDR(l)) {
        res = STk_eval(CAR(l), env);
      }
    }
  }
  return res;
}

static PRIMITIVE unless(SCM l, SCM env, int argcount)
{
  SCM res = UNDEFINED;

  if (argcount > 1) {
    if (STk_eval(CAR(l), env) == Ntruth) {
      for (l = CDR(l); !NULLP(l); l = CDR(l)) {
        res = STk_eval(CAR(l), env);
      }
    }
  }
  return res;
}

PRIMITIVE STk_init_when_unless(void)
{
  add_new_primitive("when",        tc_fsubr, when);
  add_new_primitive("unless",      tc_fsubr, unless);
  return UNDEFINED;
}
\end{verbatim}
\end{quote}
{\caption{Source listing of file {\tt when\_unless.c}}}
\label{when}
\vskip2mm\hrule\vskip3mm
\end{figure}

\subsection{Signaling errors}
For now, only one function is provided to signal errors: \Indextt{STk\_err}. This
function takes two parameters: a C string which constitutes the body of the
message and a Scheme object (a {\tt SCM} pointer) designating the {\em
erroneous} object.  If the second argument is \Indextt{NIL}, it will not be
printed. Execution of the function {\tt STk\_err} never returns.  It provokes a
jump at the start of the top-level loop. Hereafter, is a new implementation of
the {\tt when} function which uses {\tt STk\_err} when given an erroneous
arguments list.
\begin{Code}
\begin{alltt}
static PRIMITIVE when(SCM l, SCM env, int argcount)
\{
  SCM res = UNDEFINED;

  switch (argcount) \{
    case 0:  STk_err("when: no argument list given", NIL);
    case 1:  STk_err("when: null body", NIL);
    default: {\em /* Argument list is well formed.
              * Evaluate each expression of the body 
              */}
             if (STk_eval(CAR(l), env) != Ntruth)
               for (l = CDR(l); !NULLP(l); l = CDR(l)) 
                 res = STk_eval(CAR(l), env);
  \}
  return res;
\}
\end{alltt}
\end{Code}

\section{Variables}

This section shows how you can access a Scheme variable within C code. It also
shows how you can connect a Scheme and C variable such that modifying it in
Scheme will modify the associated variable and {\em vice versa}.

\subsection{Scheme Symbols and Variables}

\begin{Lentry}
\item[Defining a symbol]
Interning a symbol in the global table of symbols is done with the
\Indextt{STk\_intern} C function. Since this function is often used, you can use
the C macro \Indextt{Intern} as a shortcut. The result of Intern is the {\tt SCM}
object which denotes the scheme symbol associated to the C string passed as parameter.
For example, assigning the list 
\begin{Code}
\begin{listing}[200]{2}
'(green orange red)
\end{listing}
\end{Code}
to the C variable {\tt fire} can be done by
\begin{Code}
\begin{listing}[200]{2}
SCM fire =  Cons(Intern("green"),
                 Cons(Intern("orange"),
                      Cons(Intern("red"), NIL)));
\end{listing}
\end{Code}
Since this notation is difficult to read, some macros have been defined in
\Indextt{Src/stk.h} for building list. These macros are called \Indextt{LISTx} where
x is a number (comprised between 1 and 9) which represent the length of the list
to create. Thus, the previous example could have been written as 
\begin{Code}
\begin{listing}[200]{2}
SCM fire =  LIST3(Intern("green"), Intern("orange"), Intern("red"));
\end{listing}
\end{Code}

\item[Reading a variable]
Reading a variable in Scheme corresponds in fact to look at the value associated
to a symbol. The value associated to a symbol can be obtained with the 
\Indextt{STk\_get\_symbol\_value} C macro. This macro returns a {\tt SCM} object
which correspond to the value associated to the symbol whose name is equal to
the parameter string. \Indextt{STk\_get\_symbol\_value} returns the special
value \Indextt{UNBOUND} is this symbols has no value in the global
environment. The following piece of code
\begin{Code}
\begin{listing}[200]{2}
{
  SCM val = STk_get_symbol_value("foo");

  if (val == UNBOUND)
    STk_err("foo is undefined", NIL);
  else
    STk_display(val, UNBOUND);
}
\end{listing}
\end{Code}
displays the value of the {\tt foo} symbol, or a message is {\tt foo} is undefined in the
global environment. Note the use of the \Indextt{STk\_display} function which
implement the behavior of the Scheme {\tt display} primitive. This call correspond to
a call to {\tt display} with only one parameter, since second parameter is set
to {\tt UNBOUND} (output is done on the standard output port in this case).

\item[Setting a variable]
Setting a Scheme variable corresponds to associate a new value to a symbol. The
value of a symbol can be set with the \Indextt{STk\_set\_symbol\_value} C macro.
For example, 
\begin{Code}
\begin{listing}[200]{2}
STk_set_symbol_value("bar", STk_makeinteger(3L));
\end{listing}
\end{Code}
sets the value of the {\tt bar} symbol to the integer 3. Note that you can set a
symbol in C without using a {\tt define} form as it is necessary in Scheme.
\end{Lentry}

\subsection{Connecting Scheme and C variables}

When building a specialized interpreter, it could be useful to have a variable
you can access both in Scheme an in C. Modifying such a variable in C must
modify the Scheme associated variable and, symmetrically, modifying it in Scheme
must modify the corresponding C variable. One way to do this connection consists
to create a special Scheme variable whose content is read/written by a special
getter/setter. Definition of such a variable, is done by calling the function
\Indextt{STk\_define\_C\_variable}. The C prototype for this function is
\begin{Code}
\begin{listing}[200]{2}
void STk_define_C_variable(char *var, 
                           SCM  (*getter)(char *var), 
                           void (*setter)(char *var, SCM value));
\end{listing}
\end{Code}

The following piece of code shows how we can connect the Scheme variable {\tt *errno*} to the C
variable {\tt errno}:

\begin{Code}
\begin{listing}[200]{2}
static SCM get_errno(char *s)
{
  return STk_makeinteger((long) errno);
}
\end{listing}
\end{Code}
\begin{Code}
\begin{listing}[200]{2}
static void set_errno(char *s, SCM value)
{
  long n = STk_integer_value_no_overflow(value);

  if (n == LONG_MIN) Err("setting *errno*: bad integer", value);
  errno = n;
}
\end{listing}
\end{Code}
\begin{Code}
\begin{listing}[200]{2}
{
  ...
  STk_define_C_variable("*errno*", get_errno, set_errno);
  ...
}
\end{listing}
\end{Code}

After this call to {\tt STk\_define\_C\_variable}, reading ({\em
resp}. writing) the value of the {\tt *errno*} Scheme variable calls the
{\tt get\_errno} ({\em resp}. {\tt set\_errno}) C function.

\section{Calling Scheme from C}

Sometimes, it could be necessary to execute some Scheme code from a C function.
If the Scheme function you have to call is a primitive, it is preferred to call
directly the C function which implement it. To know the name of the C function
which implement a Scheme primitive, you'll have to look in the C file {\tt
primitive\.c} which contains the list of all the primitives of the core
interpreter. If the Scheme code you want to execute is not a call to a
primitive, it is generally easier to put your code in a C string and call the C
function \Indextt{STk\_eval\_C\_string}. This function takes two parameters: the string
to evaluate and the environment in which evaluation must take place. As for {\tt
STk\_eval}, a \Indextt{NIL} value for the environment denotes the global
environment. Suppose, for instance, that you have already written in Scheme the
{\tt fact} procedure; evaluating the factorial of 10 can be done in C with:
\begin{Code}
\begin{listing}[200]{2}
STk_eval_string("(fact 10)", NIL);
\end{listing}
\end{Code}
This call returns a pointer on a Scheme object (a {\tt SCM} pointer) containing
the result of the evaluation. If an error occurs during evaluation. It is
signaled to the user and the constant NULL is returned by {\tt STk\_eval\_string}.

\section{Adding new types}

This sections discusses how to add a new type to the {\stk} interpreter. Interested
reader can find some new types definitions in the {\tt Extensions} directory of
STk. {\stklos}, in particular, is written as an extended type whose definition
is dynamically done as soon as {\em objects} are needed. Hash tables, processes
and sockets are other examples of extended types. 

\subsection{Definition of a Scheme extended type}

Defining a scheme extended type is a little bit more complicated than
defining new primitives since it implies to take into account how this
new type interact with the GC (\Index{Garbage Collector}). Note that
until now we have not discussed about GC problems since the
interpreter is able to hide you it, as far as you don't define new
types. 

To illustrate the discussion, we will show how to add the {\em stack}
type to the {\stk} interpreter in this section. The complete code for
this section can be found in appendix.\label{stack}

\subsubsection{How the GC works}

Before showing how to define a new Scheme type, it is important to understand
how the GC works. First a certain number of cells are created\footnote{by
default 20~000; Use the -cells option of the interpreter to change this
default}. When the interpreter needs a new cell, in the {\tt cons} primitive for
instance, it will take an unused cell in the pool of pre-allocated cells. If no
more cell is available in this area, the GC is called. Its works is divided in
two phases. First phase consists to mark all the cells which are currently in
use. Finding the cells which are in used is done by marking recursively all the
object which are accessible from
\begin{itemize}
    \item the Scheme symbol table,
    \item the registers used by the program,  
    \item the C stack,
    \item global variables of type {\tt SCM}.
\end{itemize}
Marking phase is recursive; that means that if a variable denotes a list, all
the elements of this list have to be marked , to avoid that the GC frees some of
them. Of course, the recursive call for marking the component of a cell depends
on the cell's type. This first phase is called the {\em \Index{marking phase}}.

The second phase of the GC is called the {\em \Index{sweeping phase}}. It is
relatively simple: each allocated cells whose mark bit is unset is placed in the
list of free cells, since nobody points anymore on it.

If no cells can be obtained when the sweeping phase terminates, the pool of
pre-allocated cells will be extended by a new bank of cells. 

\subsubsection{The Extended type data structure}

Defining a new Scheme type consists mainly to define a new
\Indextt{STk\_extended\_scheme\_type} structure and fill in this fields. This
structure is defined as:
\begin{Code}
\begin{listing}[200]{2}
typedef struct {
  char *type_name;
  int  flags;                   
  void (*gc_mark_fct)(SCM x);
  void (*gc_sweep_fct)(SCM x);
  SCM  (*apply_fct)(SCM x, SCM args, SCM env);
  void (*display_fct)(SCM x, SCM port, int mode);
} STk_extended_scheme_type;
\end{listing}
\end{Code}
Each field of this structure is defined below
\begin{Lentry}
\item[type\_name] is a string. It denotes the external name of the new type. The
purpose of this field is mainly for debugging.

\item[flags] is the union of binary constants. For now, only two constants are
defined:
    \begin{itemize}
    \item \Indextt{EXT\_ISPROC} must be set if the new type is a procedure (i.e. if the Scheme
    procedure must answer \#t when called with this object).
    \item \Indextt{EXT\_EVALPARAM} must be set if the new type must evaluates its parameters
    when used as a function.
    \end{itemize}

\item [gc\_mark\_fct] is a pointer to the function which marks objects of the
extended type. The code associated to this function is simple. It consists to
call \index{STk\_gc\_mark} on each field whose type is {\tt SCM} in the type
associated data. This function is automatically called by the interpreter when
it scans all the used cells in the GC \Index{marking phase}.
One example of {\tt gc\_mark\_fct} is given below. 

\item [gc\_sweep\_fct] is a pointer to the function which frees the resources
allocated for representing the new type of object. This function is
automatically called by the GC in the sweeping phase for each cell which is
unused. 

\item [apply\_function] is a pointer to a function which is called when applying
this object to a list of arguments.  This function can only be called if the bit
\Indextt{EXT\_ISPROC} is set. The arguments given to this function are 
evaluated if the \Indextt{EXT\_EVALPARAM} bit is set. Finally, the environment in
which the call is done is passed as the third argument of the {\tt
apply\_function}. It serves principally when the {\tt EXT\_EVALPARAM} bit is
unset.

Set the {\tt apply\_function} to NULL to use the interpreter default apply
function. The default function raises an error when called. You can use the
default function when the new type you define is not a function.

\item [display\_fct] is a pointer to a C function which displays objects of the
new type. The display function has three parameters. The first parameter is the
object to print. The second parameter is the port to which the object must be
printed. Printing an object must be done with one of the following functions
{\tt
\begin{itemize}
\item int  \Index{STk\_getc}(FILE *f);
\item int  \Index{STk\_ungetc}(int c, FILE *f);
\item int  \Index{STk\_putc}(int c, FILE *f);
\item int  \Index{STk\_puts}(char *s, FILE *f);
\item int  \Index{STk\_eof}(FILE *f);
\end{itemize}
}
Those functions are extensions of their C equivalent: they are able to handle the
{\stk} string ports.

The third parameter of the {\tt display\_fct} is a mode constant which can take three
different values. 
\begin{itemize}
\item \Indextt{DSP\_MODE} is used when the object must be {\em displayed} in a
human readable format (as with {\tt display});
\item \Indextt{WRT\_MODE} is used when the object must be {\em written} in a
machine readable format (as with {\tt write});
\item \Indextt{TK\_MODE} is used when the object must be passed to a Tk
command. This permits to customize the way a Scheme object is converted to a
string when discussing with the Tk library.
\end{itemize}
\paragraph*{Note:} {\tt  display\_fct} can be set to NULL. In this
case the interpreter uses a default printing format. This default
format print the name of the type (found in the {\tt type\_name})
followed by an hexadecimal address.
\end{Lentry}
\subsubsection{Registering the new type}

Once a {\tt STk\_extended\_scheme\_type} structure is defined, the new type can
be registered into the interpreter. Registering a new type is done by the 
\Indextt{STk\_add\_new\_type} function. 
The prototype of this function is given below
\begin{Code}
\begin{listing}[200]{2}
int STk_add_new_type(STk_extended_scheme_type *p)
\end{listing}
\end{Code}
The integer returned by this function is the (unique) key associated to the new
type. This key is stored in each cell of the new type.

We have now enough material to define the {\tt
STk\_extended\_scheme\_type} for the new type {\em stack}. This
declaration can be done in the following way:
\begin{Code}
\begin{listing}[200]{2}
static void mark_stack(SCM p);
static void free_stack(SCM p);
static void display_stack(SCM s, SCM port, int mode);
static int tc_stack;
\end{listing}
\end{Code}
\begin{Code}
\begin{listing}[200]{2}
static STk_extended_scheme_type stack_type = {
  "stack",              /* name */
  0,                    /* is_procp */
  mark_stack,           /* gc_mark_fct */
  free_stack,           /* gc_sweep_fct */
  NULL,                 /* apply_fct */
  display_stack         /* display_fct */
};
\end{listing}
\end{Code}

This definition tells the interpreter that the new type is not a
procedure (field  {\tt is\_procp} is set to 0). Consequently, the 
{\tt apply\_fct} is set to NULL. Note that a display function is
provided here. It permits to used a customized printing
function. 

\subsubsection{New type instances creation}

Creation of a new instance of the extended type necessitates the
definition of a constructor function. This constructor obeys always
the same framework. First you have to create a new cell with the
\Indextt{NEWCELL} macro. This macro has two parameters, a {\tt SCM}
object which will point the new cell and the type of the cell to
create. The second argument is generally equal to the value returned by 
\Indextt{STk\_add\_new\_type}. 
Once the cell is created, we have generally to (dynamically) allocate
a C structure which contains the informations which are necessary to
implement the new type. Dynamic allocation  can be done with the
function \Indextt{STk\_must\_malloc}. The area returned by 
\Indextt{STk\_must\_malloc} must be stored in the {\tt data} field of
the new cell. This field can be accessed with the \Indextt{EXTDATA}
macro. 

Let's go back to the stack example. We can now define a new primitive
function to make a new stack. Provided that the global variable {\tt tc\_stack}
already contains the value returned by {\tt STk\_add\_new\_type}, we
can write

\begin{Code}
\begin{listing}[200]{2}
#define STACKP(x)       (TYPEP(x, tc_stack))
#define NSTACKP(x)      (NTYPEP(x, tc_stack))
#define STACK(x)        ((Stack *) EXTDATA(x))

typedef struct {
  int len;
  SCM values;
} Stack;
\end{listing}
\end{Code}
\begin{Code}
\begin{listing}[200]{2}
static PRIMITIVE make_stack(void)
{
  SCM z;

  NEWCELL(z, tc_stack);
  EXTDATA(z)       = STk_must_malloc(sizeof(Stack));
  STACK(z)->len    = 0;
  STACK(z)->values = NIL;
  return z;
}
\end{listing}
\end{Code}

Here, the {\tt Stack} structure is used to represent a stack. This
structure contains two fields: {\tt len} and {\tt values}. Since the latter
field is a {\tt SCM} object, it must be recursively marked when a
stack is marked. We can now define the utility function necessary for
the GC:
\begin{Code}
\begin{listing}[200]{2}
static void mark_stack(SCM p)
{
  STk_gc_mark(STACK(p)->values);
}

static void free_stack(SCM p)
{
  free(EXTDATA(p));
}
\end{listing}
\end{Code}

To terminate with this example, we give below the code of the primitive 
{\tt stack\_push!}. Other primitive are built in the same fashion and
will not be described here. A complete listing of the stack
implementation is given in appendix.
\begin{Code}
\begin{listing}[200]{2}
static PRIMITIVE stack_push(SCM s, SCM val)
{
  Stack *sp;

  if (NSTACKP(s)) STk_err("stack-push: bad stack", s);

  sp         =  STACK(s);
  sp->len   += 1;
  sp->values = Cons(val, sp->values);
  
  return UNDEFINED;
}
\end{listing}
\end{Code}

\subsection{Definition of a C extended type}

The {\stk} interpreter permits to handle C pointers as first class
objects. 
[for eg: find an example to explain how it works: gdbm?]

\subsection{About memory: Common pitfalls}


\section{Loading an extension}

{\stk} support dynamic loading for several architectures/systems. The
way to provide dynamic loadable modules is different from one system
to another and you will have to adapt what is said here to the
conventions used by your system, architecture or compiler. 
Static loading can be used for systems which doesn' support dynamic
loading (such as Ultrix) or for which the interpreter doesn't support
yet dynamic loading.

\paragraph*{Note:} {\stk) also supports the \Index{DLD Gnu package} for
dynamic loading. DlD is a library package of C functions that performs
"dynamic link editing".  Since the time to load dynamically a module
with this package is rather long, it is preferred to avoid to use it.
However, this package is the only way to provide dynamic loading on
Linux systems which don't support the \Index{ELF} format (versions 1.0
to 1.2). Since the ELF format is becoming the new standard for Linux,
this package will be no more necessary in the future.

The last version of the DLD package can be found at several places:
{\tt
\begin{itemize}
\item ftp-swiss.ai.mit.edu:pub/scm
\item prep.ai.mit.edu:/pub/gnu/jacal
\item ftp.cs.indiana.edu:/pub/scheme-repository/imp/SCM-support
\end{itemize}
}

We suppose here that we want to include the {\tt posix} module defined
in section\ref{simple-example} into the {\stk} interpreter. 
\begin{Lentry}

\item [Dynamic Loading]
If the system running {\stk} supports dynamic loading (and if the
interpreter has been compiled with dynamic loading support), you
compile your source file file to make a {\em \Index{shared object}}
file. On SunOs~4.1, for instance, this can be done by compiling the
module with the {\em \Index{pic compilation option}} (pic stands for 
{\em position independent code}). Once compilation is done, you can
pre-load your file with the line
\begin{Code}
\begin{listing}[200]{2}
ld -assert pure-text -o time.so time.o
\end{listing}
\end{Code}
This will produce a file name {\tt posix.so} which can be loaded with the 
\Indextt{load} Scheme primitive procedure. The {\tt load} primitive recognizes
that this file is a shared object and calls a function whose
name the concatenation of the string 
{\tt STk\_init\_}\index{STk\_init\_ prefix} and the base name of file
loaded. Thus, loading the file {\tt posix.so} impies the call of a
pimitive whose name is {\tt STk\_init\_posix}. 

Look at {\tt Src/Extensions} directory to see some examples of shared
object construction.

\paragraph*{Note:} when the {\stk} is built, the \Indextt{Makefile}  in
the {\tt Src/Extensions} is customized for your system/compiler. A
simple way to determine the options you have to use for compiling your
program consist to run the \Indextt{make} command on one of the file present in
this directory. For instance, issuing the following command
\begin{Code}
\begin{listing}[200]{2}
make -n posix.so
\end{listing}
\end{Code}
on a Linux box using the DLD package will output the following lines:
\begin{Code}
\begin{listing}[200]{2}
gcc  -g -DSTk_CODE -DUSE_DLD -DLINUX   -DHAVE_UNISTD_H=1 \
        -DHAVE_SIGACTION=1 -I../Tk -I../Tcl -I../Src -I../Mp \
        -I/usr/X11R6/include  -c posix.c -o posix.o
ld -r -o posix.so posix.o
\end{listing}
\end{Code}

\item [Static loading]
A C module which define a new type can also be statically loded in the
interepreter. To load your module, you have to modify the {\tt Src/Makefile}
or {\tt  Snow/Makefile}. Once yo have added your extension object in
the \Indextt{USER\_OBJ} variable , you must modify the file
\Indextt{Src/userinit.c} to add you initialization (and eventually
cleanup) code. The call to your initialization function must be done in
the \Indextt{STk\_user\_init} C function. Once this is done, you can
run the {\tt make} command again to build the extended interpreter.
\end{Lentry}

\section{Adding new Tk widgets}

\subsection{Widget compilation}

Adding a new Tk widget to the {\stk} interpreter is generally a simple
{\em hack}. Most of the time, extension widgets written
for Tcl/Tk can be added to the {\stk} interpreter without modifying
the source code of the widget. However, there is no unique method to
add a widget to the Tcl interpreter; consequently, what is given below
is a set of hints to widget integration rather than a {\em always
working recipe}. To illustrate this section, we will see how we can add
the {\em fscale} widget (a floating-point scale widget available on the
Tcl/Tk repository in the {\tt tkFScale-?.?.tar.gz} file)\footnote{
This widget is now integrated in the standard Tk4.0; it is provided as
an extension widget with Tk3.6.}.

Generally, the code of a Tcl/Tk extension widget can be divided in two
parts: the code which implement widget's behavior and the extension
initialization code. Extension initialization code, in Tcl/Tk, must be
placed in the procedure {\tt Tcl\_AppInit} which is located in the file
{\tt tkAppInit.c}. If the extension package adds a lot of widgets, it
generally defines a function to do all the initializations. On the
other hand, if the extension only defines a single widget, the
extension code generally consists to call the C function 
\Indextt{Tcl\_CreateCommand} for each new widget defined in the
extension. {\tt Tcl\_CreateCommand} is the Tcl standard way to add a
new command. This function also exists in the {\stk} interpreter; it
creates a new \Index{Tk command} object~\cite{Gallesio95-1}. The
prototype of this function is:
\begin{Code}
\begin{listing}[200]{2}
void Tcl_CreateCommand(Tcl_Interp *interp,
                       char *cmdName, 
                       Tcl_CmdProc *proc,
                       ClientData clientData,
                       Tcl_CmdDeleteProc *deleteProc));
\end{listing}
\end{Code}

For {\stk},
\begin{itemize}
\item the {\tt interp} is always the global variable
                \Indextt{Stk\_main\_interp}
\item {\tt cmdName} is the name of the widget in the Scheme world.
\item {\tt proc} is the name of the C function which implement the {\em Tk command}
\item {\tt clientData} are informations which are associated to the
widget code. For a new widget, {\tt clientData} can generally set to 
the result of 
\begin{Code}
\begin{listing}[200]{2}
Tk_MainWindow(Stk_main_interp);
\end{listing}
\end{Code}
\item {\tt deleteProc} is a function which is called when the widget
is destroyed. You generally don't need to change the value of this
parameter (which is often set to {\tt NULL}). 
\end{itemize}

The usual way to integrate this initialization code in a Tcl interpreter
consists to patch the {\tt tkAppInit.c} file to add the call to the
initialization (or the {\tt Tcl\_CreateCommand}) function.
To add an extension written for Tcl/Tk to {\stk}, all that is needed
consists to adapt the initialization code for {\stk}. For example, the
{\tt fscale} widget initialization code adds the following call in the
body of the {\tt Tcl\_AppInit} function:
\begin{Code}
\begin{listing}[200]{2}
Tcl_CreateCommand(interp, "fscale", Tk_FScaleCmd, 
                  (ClientData) main, 
                  (void (*)()) NULL);
\end{listing}
\end{Code}

For {\stk}, this call can be written 
\begin{Code}
\begin{listing}[200]{2}
Tcl_CreateCommand(STk_main_interp, "fscale", Tk_FScaleCmd, 
                  (ClientData) Tk_MainWindow(STk_main_interp), 
                  (void (*)()) NULL);
\end{listing}
\end{Code}

This call must executed before trying to create a new {\tt fscale}
widget.

\subsection{Widget linking}

The cleaner way to add a new widget to {\stk} consists to define a
special C module for this widget. Defining the widget in a C module
allows us to make the new widget dynamically loadable. The code for 
making the {\tt fscale} widget dynamically loadable could be:

\begin{Code}
\begin{listing}[200]{2}
/* Contents of the file fscale.c */
#ifndef USE_TK
#define USE_TK
#endif
\end{listing}
\end{Code}
\begin{Code}
\begin{listing}[200]{2}
#include <stk.h>
/* 
 *include the widget source code. Ugly but this avoid to have two
 * source files to link
 */
#include "tkFScale.c"
\end{listing}
\end{Code}
\begin{Code}
\begin{listing}[200]{2}
PRIMITIVE STk_init_fscale(void)
{
  Tcl_CreateCommand(STk_main_interp, 
                    "fscale", 
                    Tk_FScaleCmd, 
                    (ClientData) Tk_MainWindow(STk_main_interp), 
                    (void (*)()) NULL);
}
\end{listing}
\end{Code}

\section{Extending the interpreter with C++}

[for eg : Identify the problems]

\section{Embedding the STk interpreter}

[for eg: This parts need some work in the interpreter]



%%%%%%%%%%%%%%%%%%%%%%%%%%%%%%%%%%%%%%%%%%%%%%%%%%%%%%%%%%%%%%%%%%%%%%%%%%%%%%

\pagebreak
{\Large\bf Appendix}
\par
Hereafter is the complete code for the stack type discussed in \ref{stack}
{\footnotesize
\verbatiminput{stack.c}
}

%%%%%%%%%%%%%%%%%%%%%%%%%%%%%%%%%%%%%%%%%%%%%%%%%%%%%%%%%%%%%%%%%%%%%%%%%%%%%%
\bibliography{bibliography}
%
% A note on how to extend the STk interpreter
%
% Copyright � 1993-1999 Erick Gallesio - I3S-CNRS/ESSI <eg@unice.fr>
%
% Permission to use, copy, modify, distribute,and license this
% software and its documentation for any purpose is hereby granted,
% provided that existing copyright notices are retained in all
% copies and that this notice is included verbatim in any
% distributions.  No written agreement, license, or royalty fee is
% required for any of the authorized uses.
% This software is provided ``AS IS'' without express or implied
% warranty.
%
%           Author: Erick Gallesio [eg@unice.fr]
%    Creation date:  in 1993
% Last file update:  3-Sep-1999 19:36 (eg)
%

\documentclass[10pt]{article}
\usepackage{a4wide}
\usepackage{fancyheadings}
\usepackage{fancybox}
\usepackage{eg-commands}

\pagestyle{fancyplain}
\makeindex\parindent0pt\parskip2mm
\begin{document}
\bibliographystyle{plain}

\title{Extending the ST{\large\bf{K}} interpreter}
\author{Erick Gallesio \\
Universit\'e de Nice~~-~~Sophia-Antipolis \\
Laboratoire I3S - CNRS URA 1376 - ESSI. \\
Route des Colles\\
B.P. 145\\
06903 Sophia-Antipolis Cedex - FRANCE\\[3mm]
email: eg@unice.fr}
\date{July 1995}
\maketitle

\begin{abstract}
This document describes how to extend the {\stk} interpreter with new
primitives procedures and/or new types. Extending the interpreter can be done
by writing new {\em modules} in C. New C code can be statically linked to the
core interpreter or dynamically loaded on operating systems which support
shared libraries. This document also presents how to integrate new
Tk widgets written for the Tcl interpreter in {\stk}.
\end{abstract}


\pagebreak
\tableofcontents
\pagebreak
\section{Introduction}
This document describes how to extend the {\stk}\cite{Gallesio93-1}
interpreter using the C language\cite{Kernighan:CPL88}. To begin, we
will start with a simple extension which will only consist to add some
simple new primitives to the interpreter. Second section will describe
how to add a new type (and the primitives for manipulating this new
type). Another interesting extension consists to add new kind of
primitives (i.e.  primitives which evaluate their argument in
particular way). This kind of extension will be discussed in the third
section.  Fourth section discusses how to add a new widget to the
interpreter. Calling some Scheme code from a C function is showed in
section 5. And last, we will show how to load an extension at load
time. This facility will permit to extend the {\stk} interpreter
without having to recompile it, on systems which support dynamic
loading.

\section{Adding new primitives}

\subsection{A simple example}

\label{simple-example}
One of the simpler extension one can wish to do consists to add new
primitives procedures to the interpreter. To illustrate this, suppose
we want to add two new primitives to the interpreter: {\tt posix-time}
and {\tt posix-ctime}.  The former function correspond to the
POSIX.1\cite{POSIX.1-90} function {\tt time}: it returns the number
of seconds elapsed since 00:00:00 on January 1, 1970, Coordinated
Universal Time (UTC).  The latter is a wrapper around the POSIX.1
function {\tt ctime} which returns a string containing the current
time in an human readable format.

First, we will see how to write the new Scheme primitive {\tt posix-time}.
Implementing a new primitive requires to write a new C function which will do
the work. Here, we write the C function {\tt posix\_time} to implement the
Scheme primitive {\tt posix-time}. The code of this function is given below.
\begin{Code}
\begin{listing}[200]{2}
static PRIMITIVE posix_time(void)
{
  return STk_makeinteger((long) time(NULL));
}
\end{listing}
\end{Code}
This function uses the interpreter \Indextt{STk\_makeinteger} function
which converts a C long integer to a {\stk} integer. Once the {\tt
posix\_time} C function is written, we have to bind this new primitive to
the Scheme symbol {\tt posix-time}. This is achieved by the following C
function call.
\begin{Code}
\begin{listing}[200]{2}
STk_add_new_primitive("posix-time", tc_subr_0, posix_time);
\end{listing}
\end{Code}

\paragraph*{Note:} The C type \Indextt{SCM} is used to describe the objects
manipulated in Scheme. \Indextt{PRIMITIVE} is an alias for this type; it
is preferably used when defining a new primitive.

\Indextt{STk\_add\_new\_primitive} tells the interpreter that the Scheme 
symbol {\tt posix-time} must be bound to the (C written) primitive
{\tt posix\_time}.  The constant \Indextt{tc\_subr\_0} used as the second
argument indicates the arity of this primitive. In this case, the
arity of the primitive is 0.

Let's now have a look at the primitive {\tt posix-ctime}. A first
writing of this primitive could be
\begin{Code}
\begin{listing}[200]{2}
static PRIMITIVE posix_ctime(void)
{
  char *s;
  time_t t = time(NULL);

  s = ctime(&t);
  return STk_makestring(s);
}
\end{listing}
\end{Code}

This functions uses another interpreter routine (\Indextt{STk\_makestring})
which takes as parameter a null terminated string and returns a
Scheme string. 

Binding of the scheme symbol {\tt time-string} to the C function 
{\tt get\_time} is done by the call 
\begin{Code}
\begin{listing}[200]{2}
STk_add_new_primitive("posix-ctime", tc_subr_0, posix_ctime);
\end{listing}
\end{Code}

A complete listing of this code is given in Figure~\ref{posix-1}.
Provided that we have done a shared object of this file, and that its
name is {\tt posix.so}, our two new primitives can be loaded
dynamically by:
\begin{Code}
\begin{listing}[200]{2}
(load "time.so")
\end{listing}
\end{Code}

\paragraph*{Notes:}
\begin{itemize}
\item Suffix can be omitted. Suffixes given in the Scheme variable 
\Indextt{*load-suffixes*} gives the order in which suffixes must be
tried for loading a file.  Default value for this variable is
{\tt ("stk" "stklos" "scm" "so")}.

\item When dynamic loading is used, the interpreter try to call a 
function whose name is equal to the string {\tt "STk\_init\_"} followed by the
name of the file, without suffix. Definitions of new primitives are generally done
in this function. Here, the C function in charge of module initialization must
be called {\tt STk\_init\_posix}.
\end{itemize}

%%%%%%%%%%%%%%%%%%%%%%%%%%%%%%%%%%%%%%%%%%%%%%%%%%%%%%%%%%%%%%%%%%%%%%%%%%%%%%
\begin{figure}
\begin{quote} \footnotesize
\begin{alltt}
#include <sys/types.h>
#include <sys/time.h>
#include <time.h>
#include <stk.h>        {\em /* Declaration of STk objects/primitives */}

static PRIMITIVE posix_time(void)
\{
  return STk_makeinteger((long) time(NULL));
\}

static PRIMITIVE posix_ctime(void)
\{
  char *s;
  time_t t = time(NULL);

  s = ctime(&t);
  return STk_makestring(s);
\}

PRIMITIVE STk_init_posix(void)
\{
  STk_add_new_primitive("posix-time",  tc_subr_0, posix_time);
  STk_add_new_primitive("posix-ctime", tc_subr_0, posix_ctime);
  return UNDEFINED;
\}
\end{alltt}
{\caption{A first version of file {\tt posix.c}}}
\label{posix-1}
\vskip2mm\hrule\vskip3mm
\end{quote}
\end{figure} 

\subsection{Passing arguments to a primitive}

This section shows how to pass arguments to a new primitive written in
C. To illustrate our purpose, we will rewrite the primitive {\tt
posix-ctime} to be conform to POSIX.1 (this function should take an
integer, a count of seconds, and should return corresponding date as a
string). A second writing of previous function could be:

\begin{Code}
\begin{listing}[200]{2}
static PRIMITIVE posix_ctime(SCM seconds)
{
  long sec;

  sec = STk_integer_value_no_overflow(seconds);
  return STk_makestring(ctime((time_t *)&sec));
}
\end{listing}
\end{Code}


This function has one parameter since Scheme primitive arity is one. The
C primitives parameters are always \Indextt{SCM} objects. An object of this type is
a pointer to a \Indextt{struct obj}: the type which permits to represent all the
Scheme objects.  The {\tt SCM} and {\tt struct~obj} types definitions can
be found in the \Indextt{Src/stk.h} header file. 

The first job of this function consists to convert the Scheme
parameter ({\tt seconds}) to a C integer {\tt long int}. This is done with the
function \Indextt{STk\_integer\_value\_no\_overflow}which takes a {\tt SCM}
and returns a {\tt long int}. This functions returns {\tt LONG\_MIN} if
the argument is not a an integer number (or a number which don't fit in the C
representation of a C {\tt long int}). Once this conversion is done,
the rest of the job is similar to the code presented above.

To add this primitive to the global Scheme environment, we have to
change the previous {\tt STk\_add\_new\_primitive} for this primitive by:

\begin{Code}
\begin{listing}[200]{2}
STk_add_new_primitive("posix-ctime", tc_subr_1, posix_ctime);
\end{listing}
\end{Code}

in the init section. This call states that the type of this primitive
is fixed to a \Indextt{tc\_subr\_1} (a arity-1 primitive).

However, this function is not too satisfying, even if close to the
POSIX definition: it obliges to pass a parameter which will be
probably most of the time the result of the primitive {\tt
posix-time} (i.e. the most frequent usage of this function will be

\begin{Code}
\begin{listing}[200]{2}
(posix-ctime (posix-time))
\end{listing}
\end{Code}

which is not very elegant). A better approach consists to allow this
primitive to have a optional parameter. This permits to be at the
same time conform to the POSIX convention and close to Scheme habits. 
The following version implements the {\tt posix-ctime} with a optional
parameter:
\begin{Code}
\begin{listing}[200]{2}
static PRIMITIVE posix_ctime(SCM seconds)
{
  long sec;

  sec = (seconds==UNBOUND) ? time(NULL)
                           : STk_integer_value_no_overflow(seconds);
  return STk_makestring(ctime((time_t *) &sec));
}
\end{listing}
\end{Code}

If the Scheme {\tt posix-ctime} primitive is called with one parameter, it
will be passed to the C function in the {\tt seconds} parameter. If {\tt
posix-ctime} is called without parameter, {\tt seconds} is set to the special
value \Indextt{UNBOUND}. So, the first test in this function consists to set a
correct value to the variable {\tt sec}; this value is either the current
time, either the given integer, depending of the number of parameters
given to {\tt posix-ctime}.

Of course, the type of this new primitive must be changed to allow 0 or 1
parameter. This is done by changing the {\tt tc\_subr\_0} in the previous 
{\tt STk\_add\_new\_primitive} by \Indextt{tc\_subr\_0\_or\_1}.

The following types are available for C primitives:
\begin{itemize}
\item \Indextt{tc\_subr\_0} for arity-0 primitives
\item \Indextt{tc\_subr\_1} for arity-1 primitives
\item \Indextt{tc\_subr\_2} for arity-2 primitives
\item \Indextt{tc\_subr\_3} for arity-3 primitives
\item \Indextt{tc\_subr\_0\_or\_1} for primitives 
      which have 0 or 1 parameter
      (e.g.~{\tt read}). On the C side you have to declare a function
      which takes one {\tt SCM} argument. This argument is set to the
      (evaluated) parameter if present, to {\tt UNBOUND} otherwise.
\item \Indextt{tc\_subr\_1\_or\_2} for 
      primitives which have 1 or 2 parameters
      (e.g.~{\tt write}). Here you have to declare a C function with
      two {\tt SCM} parameters. The first one will contain the first Scheme 
      argument and the second will contain the second argument value
      or {\tt UNBOUND} if omitted.
\item \Indextt{tc\_subr\_2\_or\_3} for 
      primitives which have 2 or 3 parameters
      (there's no primitive of this type in core interpreter).
      Of course, you'll have to declare a C function with three {\tt SCM}
      parameters. Apart that, conventions are the same has before.
\item \Indextt{tc\_lsubr} for primitives which have a variable number of
      arguments. Actuals arguments are collected in a list which is given as the
      first argument of the C primitive. The second argument of the C function
      is an integer counting the actual number of arguments given to the
      primitive. Hence, the signature of the C function which implement a {\tt
      tc\_lsubr} must be
\begin{quote}
\begin{verbatim}
PRIMITIVE function(SCM arglist, int argcount);
\end{verbatim}
\end{quote}
      Note that all the Scheme arguments are evaluated during the
      construction of the list which is passed to the C function.
\item \Indextt{tc\_fsubr} is similar to {\tt tc\_lsubr} except that
      arguments are not evaluated. On the C side, you have to declare a function
      with three {\tt SCM} parameters: the list of (non evaluated) arguments,
      the current environment and the length of the arguments list. The
      signature of the C function which implement a {\tt tc\_fsubr} must be
\begin{quote}
\begin{verbatim}
PRIMITIVE function(SCM arglist, SCM env, int argcount);
\end{verbatim}
\end{quote}
      See \ref{evaluating-args} more details about {\tt tc\_fsubr}.

\item {\tt tc\_tkcommand} for primitives which follow  the Tcl
      command argument passing style (i.e.~{\em \`a la} {\tt argc/argv}). This
      is this kind of procedure that will be used for to add new widgets in the
      {\stk} interpreter. See \ref{new-widget} and \cite{ouster-book} for more
      details.
\end{itemize}

To illustrate how to write a {\tt tc\_lsubr} primitive, let's have a look at the
code, given below, of the function which implement the Scheme primitive {\tt
vector}:
\begin{quote}{\small
\begin{verbatim}
PRIMITIVE STk_vector(SCM arglist, int argcount)
{
  int j;
  SCM z = STk_makevect(argcount, NULL);
  
  for (j = 0; j < argcount; j++, arglist=CDR(arglist)) {
    VECT(z)[j] = CAR(arglist);
  }
  return z;
}
\end{verbatim}
}\end{quote}

This function receives the values passed to the {\tt vector} primitives in the
list arglist (the length of this list is stored in {\tt argcount}). This
function uses \Indextt{STk\_makevect} which returns a Scheme vector. Its first
argument is the length of the vector and its second argument is the initial
value of the vector's elements.  Next section will show how to implement a
primitive which evaluates itself its parameters (i.e. a {\tt tc\_fsubr}
primitive.

\subsection{Evaluating arguments}

In some circumstances it could be useful to add new primitives which don't
evaluate their arguments. This permits to add new control structures to the
interpreter. To illustrate this, we will add two new primitives to the {\stk}
interpreter: {\tt when} and {\tt unless}. As explained in the preceding
section, the C functions which will implement those control structures must be
of type \Indextt{tc\_fsubr}. A {\tt tc\_fsubr} primitive, on the C
side, is given three parameters when called: 
\begin{enumerate}
\item a list of its (non evaluated) parameters, 
\item the local environment when it was called (and in which
evaluations should generally take place),
\item the length of the parameters list.
\end{enumerate}

The C function can step through its parameter list using the C macros
{\tt CAR}, {\tt CDR} and {\tt NULLP} (which do the obvious work) and evaluates elements
of this list as needed. Evaluation of an expression can be done with
the \Indextt{STk\_eval} C function. {\tt STk\_eval} takes two parameters: the
expression to evaluate and the environment in which evaluation takes
place (the \Indextt{NIL} variable, by convention, denotes the Index{global environment}).

\paragraph*{Note:} {\bf a list of arguments is always a proper list. You don't need to test
if it is well formed.}

Hereafter is the code of the {\tt when} primitive.
\begin{Code}
\begin{listing}[200]{2}
static PRIMITIVE when(SCM l, SCM env, int argcount)
{
  SCM res = UNDEFINED;

  if (argcount > 1) {
    if (STk_eval(CAR(l), env) != Ntruth) {
      for (l = CDR(l); !NULLP(l); l = CDR(l)) {
        res = STk_eval(CAR(l), env);
      }
    }
  }
  return res;
}
\end{listing}
\end{Code}

\noindent
Some points to note here:
\begin{itemize}
\item \Indextt{UNDEFINED} is an interpreter {\em constant}. It serves to denote
the notion of {\em ``\Index{unspecified result}''} of {\rrrr}.
\item \Indextt{Truth} and \Indextt{Ntruth} are two global {\em constants} of the
interpreter which denote respectively the \Indextt{\#t} and \Indextt{\#f} Scheme
constants.
\end{itemize}

\noindent
Figure~\ref{when} shows a complete implementation of {\tt when} and {\tt unless}.

\begin{figure}
\begin{quote}\footnotesize
\begin{verbatim}
#include <stk.h>

static PRIMITIVE when(SCM l, SCM env, int argcount)
{
  SCM res = UNDEFINED;

  if (argcount > 1) {
    if (STk_eval(CAR(l), env) != Ntruth) {
      for (l = CDR(l); !NULLP(l); l = CDR(l)) {
        res = STk_eval(CAR(l), env);
      }
    }
  }
  return res;
}

static PRIMITIVE unless(SCM l, SCM env, int argcount)
{
  SCM res = UNDEFINED;

  if (argcount > 1) {
    if (STk_eval(CAR(l), env) == Ntruth) {
      for (l = CDR(l); !NULLP(l); l = CDR(l)) {
        res = STk_eval(CAR(l), env);
      }
    }
  }
  return res;
}

PRIMITIVE STk_init_when_unless(void)
{
  add_new_primitive("when",        tc_fsubr, when);
  add_new_primitive("unless",      tc_fsubr, unless);
  return UNDEFINED;
}
\end{verbatim}
\end{quote}
{\caption{Source listing of file {\tt when\_unless.c}}}
\label{when}
\vskip2mm\hrule\vskip3mm
\end{figure}

\subsection{Signaling errors}
For now, only one function is provided to signal errors: \Indextt{STk\_err}. This
function takes two parameters: a C string which constitutes the body of the
message and a Scheme object (a {\tt SCM} pointer) designating the {\em
erroneous} object.  If the second argument is \Indextt{NIL}, it will not be
printed. Execution of the function {\tt STk\_err} never returns.  It provokes a
jump at the start of the top-level loop. Hereafter, is a new implementation of
the {\tt when} function which uses {\tt STk\_err} when given an erroneous
arguments list.
\begin{Code}
\begin{alltt}
static PRIMITIVE when(SCM l, SCM env, int argcount)
\{
  SCM res = UNDEFINED;

  switch (argcount) \{
    case 0:  STk_err("when: no argument list given", NIL);
    case 1:  STk_err("when: null body", NIL);
    default: {\em /* Argument list is well formed.
              * Evaluate each expression of the body 
              */}
             if (STk_eval(CAR(l), env) != Ntruth)
               for (l = CDR(l); !NULLP(l); l = CDR(l)) 
                 res = STk_eval(CAR(l), env);
  \}
  return res;
\}
\end{alltt}
\end{Code}

\section{Variables}

This section shows how you can access a Scheme variable within C code. It also
shows how you can connect a Scheme and C variable such that modifying it in
Scheme will modify the associated variable and {\em vice versa}.

\subsection{Scheme Symbols and Variables}

\begin{Lentry}
\item[Defining a symbol]
Interning a symbol in the global table of symbols is done with the
\Indextt{STk\_intern} C function. Since this function is often used, you can use
the C macro \Indextt{Intern} as a shortcut. The result of Intern is the {\tt SCM}
object which denotes the scheme symbol associated to the C string passed as parameter.
For example, assigning the list 
\begin{Code}
\begin{listing}[200]{2}
'(green orange red)
\end{listing}
\end{Code}
to the C variable {\tt fire} can be done by
\begin{Code}
\begin{listing}[200]{2}
SCM fire =  Cons(Intern("green"),
                 Cons(Intern("orange"),
                      Cons(Intern("red"), NIL)));
\end{listing}
\end{Code}
Since this notation is difficult to read, some macros have been defined in
\Indextt{Src/stk.h} for building list. These macros are called \Indextt{LISTx} where
x is a number (comprised between 1 and 9) which represent the length of the list
to create. Thus, the previous example could have been written as 
\begin{Code}
\begin{listing}[200]{2}
SCM fire =  LIST3(Intern("green"), Intern("orange"), Intern("red"));
\end{listing}
\end{Code}

\item[Reading a variable]
Reading a variable in Scheme corresponds in fact to look at the value associated
to a symbol. The value associated to a symbol can be obtained with the 
\Indextt{STk\_get\_symbol\_value} C macro. This macro returns a {\tt SCM} object
which correspond to the value associated to the symbol whose name is equal to
the parameter string. \Indextt{STk\_get\_symbol\_value} returns the special
value \Indextt{UNBOUND} is this symbols has no value in the global
environment. The following piece of code
\begin{Code}
\begin{listing}[200]{2}
{
  SCM val = STk_get_symbol_value("foo");

  if (val == UNBOUND)
    STk_err("foo is undefined", NIL);
  else
    STk_display(val, UNBOUND);
}
\end{listing}
\end{Code}
displays the value of the {\tt foo} symbol, or a message is {\tt foo} is undefined in the
global environment. Note the use of the \Indextt{STk\_display} function which
implement the behavior of the Scheme {\tt display} primitive. This call correspond to
a call to {\tt display} with only one parameter, since second parameter is set
to {\tt UNBOUND} (output is done on the standard output port in this case).

\item[Setting a variable]
Setting a Scheme variable corresponds to associate a new value to a symbol. The
value of a symbol can be set with the \Indextt{STk\_set\_symbol\_value} C macro.
For example, 
\begin{Code}
\begin{listing}[200]{2}
STk_set_symbol_value("bar", STk_makeinteger(3L));
\end{listing}
\end{Code}
sets the value of the {\tt bar} symbol to the integer 3. Note that you can set a
symbol in C without using a {\tt define} form as it is necessary in Scheme.
\end{Lentry}

\subsection{Connecting Scheme and C variables}

When building a specialized interpreter, it could be useful to have a variable
you can access both in Scheme an in C. Modifying such a variable in C must
modify the Scheme associated variable and, symmetrically, modifying it in Scheme
must modify the corresponding C variable. One way to do this connection consists
to create a special Scheme variable whose content is read/written by a special
getter/setter. Definition of such a variable, is done by calling the function
\Indextt{STk\_define\_C\_variable}. The C prototype for this function is
\begin{Code}
\begin{listing}[200]{2}
void STk_define_C_variable(char *var, 
                           SCM  (*getter)(char *var), 
                           void (*setter)(char *var, SCM value));
\end{listing}
\end{Code}

The following piece of code shows how we can connect the Scheme variable {\tt *errno*} to the C
variable {\tt errno}:

\begin{Code}
\begin{listing}[200]{2}
static SCM get_errno(char *s)
{
  return STk_makeinteger((long) errno);
}
\end{listing}
\end{Code}
\begin{Code}
\begin{listing}[200]{2}
static void set_errno(char *s, SCM value)
{
  long n = STk_integer_value_no_overflow(value);

  if (n == LONG_MIN) Err("setting *errno*: bad integer", value);
  errno = n;
}
\end{listing}
\end{Code}
\begin{Code}
\begin{listing}[200]{2}
{
  ...
  STk_define_C_variable("*errno*", get_errno, set_errno);
  ...
}
\end{listing}
\end{Code}

After this call to {\tt STk\_define\_C\_variable}, reading ({\em
resp}. writing) the value of the {\tt *errno*} Scheme variable calls the
{\tt get\_errno} ({\em resp}. {\tt set\_errno}) C function.

\section{Calling Scheme from C}

Sometimes, it could be necessary to execute some Scheme code from a C function.
If the Scheme function you have to call is a primitive, it is preferred to call
directly the C function which implement it. To know the name of the C function
which implement a Scheme primitive, you'll have to look in the C file {\tt
primitive\.c} which contains the list of all the primitives of the core
interpreter. If the Scheme code you want to execute is not a call to a
primitive, it is generally easier to put your code in a C string and call the C
function \Indextt{STk\_eval\_C\_string}. This function takes two parameters: the string
to evaluate and the environment in which evaluation must take place. As for {\tt
STk\_eval}, a \Indextt{NIL} value for the environment denotes the global
environment. Suppose, for instance, that you have already written in Scheme the
{\tt fact} procedure; evaluating the factorial of 10 can be done in C with:
\begin{Code}
\begin{listing}[200]{2}
STk_eval_string("(fact 10)", NIL);
\end{listing}
\end{Code}
This call returns a pointer on a Scheme object (a {\tt SCM} pointer) containing
the result of the evaluation. If an error occurs during evaluation. It is
signaled to the user and the constant NULL is returned by {\tt STk\_eval\_string}.

\section{Adding new types}

This sections discusses how to add a new type to the {\stk} interpreter. Interested
reader can find some new types definitions in the {\tt Extensions} directory of
STk. {\stklos}, in particular, is written as an extended type whose definition
is dynamically done as soon as {\em objects} are needed. Hash tables, processes
and sockets are other examples of extended types. 

\subsection{Definition of a Scheme extended type}

Defining a scheme extended type is a little bit more complicated than
defining new primitives since it implies to take into account how this
new type interact with the GC (\Index{Garbage Collector}). Note that
until now we have not discussed about GC problems since the
interpreter is able to hide you it, as far as you don't define new
types. 

To illustrate the discussion, we will show how to add the {\em stack}
type to the {\stk} interpreter in this section. The complete code for
this section can be found in appendix.\label{stack}

\subsubsection{How the GC works}

Before showing how to define a new Scheme type, it is important to understand
how the GC works. First a certain number of cells are created\footnote{by
default 20~000; Use the -cells option of the interpreter to change this
default}. When the interpreter needs a new cell, in the {\tt cons} primitive for
instance, it will take an unused cell in the pool of pre-allocated cells. If no
more cell is available in this area, the GC is called. Its works is divided in
two phases. First phase consists to mark all the cells which are currently in
use. Finding the cells which are in used is done by marking recursively all the
object which are accessible from
\begin{itemize}
    \item the Scheme symbol table,
    \item the registers used by the program,  
    \item the C stack,
    \item global variables of type {\tt SCM}.
\end{itemize}
Marking phase is recursive; that means that if a variable denotes a list, all
the elements of this list have to be marked , to avoid that the GC frees some of
them. Of course, the recursive call for marking the component of a cell depends
on the cell's type. This first phase is called the {\em \Index{marking phase}}.

The second phase of the GC is called the {\em \Index{sweeping phase}}. It is
relatively simple: each allocated cells whose mark bit is unset is placed in the
list of free cells, since nobody points anymore on it.

If no cells can be obtained when the sweeping phase terminates, the pool of
pre-allocated cells will be extended by a new bank of cells. 

\subsubsection{The Extended type data structure}

Defining a new Scheme type consists mainly to define a new
\Indextt{STk\_extended\_scheme\_type} structure and fill in this fields. This
structure is defined as:
\begin{Code}
\begin{listing}[200]{2}
typedef struct {
  char *type_name;
  int  flags;                   
  void (*gc_mark_fct)(SCM x);
  void (*gc_sweep_fct)(SCM x);
  SCM  (*apply_fct)(SCM x, SCM args, SCM env);
  void (*display_fct)(SCM x, SCM port, int mode);
} STk_extended_scheme_type;
\end{listing}
\end{Code}
Each field of this structure is defined below
\begin{Lentry}
\item[type\_name] is a string. It denotes the external name of the new type. The
purpose of this field is mainly for debugging.

\item[flags] is the union of binary constants. For now, only two constants are
defined:
    \begin{itemize}
    \item \Indextt{EXT\_ISPROC} must be set if the new type is a procedure (i.e. if the Scheme
    procedure must answer \#t when called with this object).
    \item \Indextt{EXT\_EVALPARAM} must be set if the new type must evaluates its parameters
    when used as a function.
    \end{itemize}

\item [gc\_mark\_fct] is a pointer to the function which marks objects of the
extended type. The code associated to this function is simple. It consists to
call \index{STk\_gc\_mark} on each field whose type is {\tt SCM} in the type
associated data. This function is automatically called by the interpreter when
it scans all the used cells in the GC \Index{marking phase}.
One example of {\tt gc\_mark\_fct} is given below. 

\item [gc\_sweep\_fct] is a pointer to the function which frees the resources
allocated for representing the new type of object. This function is
automatically called by the GC in the sweeping phase for each cell which is
unused. 

\item [apply\_function] is a pointer to a function which is called when applying
this object to a list of arguments.  This function can only be called if the bit
\Indextt{EXT\_ISPROC} is set. The arguments given to this function are 
evaluated if the \Indextt{EXT\_EVALPARAM} bit is set. Finally, the environment in
which the call is done is passed as the third argument of the {\tt
apply\_function}. It serves principally when the {\tt EXT\_EVALPARAM} bit is
unset.

Set the {\tt apply\_function} to NULL to use the interpreter default apply
function. The default function raises an error when called. You can use the
default function when the new type you define is not a function.

\item [display\_fct] is a pointer to a C function which displays objects of the
new type. The display function has three parameters. The first parameter is the
object to print. The second parameter is the port to which the object must be
printed. Printing an object must be done with one of the following functions
{\tt
\begin{itemize}
\item int  \Index{STk\_getc}(FILE *f);
\item int  \Index{STk\_ungetc}(int c, FILE *f);
\item int  \Index{STk\_putc}(int c, FILE *f);
\item int  \Index{STk\_puts}(char *s, FILE *f);
\item int  \Index{STk\_eof}(FILE *f);
\end{itemize}
}
Those functions are extensions of their C equivalent: they are able to handle the
{\stk} string ports.

The third parameter of the {\tt display\_fct} is a mode constant which can take three
different values. 
\begin{itemize}
\item \Indextt{DSP\_MODE} is used when the object must be {\em displayed} in a
human readable format (as with {\tt display});
\item \Indextt{WRT\_MODE} is used when the object must be {\em written} in a
machine readable format (as with {\tt write});
\item \Indextt{TK\_MODE} is used when the object must be passed to a Tk
command. This permits to customize the way a Scheme object is converted to a
string when discussing with the Tk library.
\end{itemize}
\paragraph*{Note:} {\tt  display\_fct} can be set to NULL. In this
case the interpreter uses a default printing format. This default
format print the name of the type (found in the {\tt type\_name})
followed by an hexadecimal address.
\end{Lentry}
\subsubsection{Registering the new type}

Once a {\tt STk\_extended\_scheme\_type} structure is defined, the new type can
be registered into the interpreter. Registering a new type is done by the 
\Indextt{STk\_add\_new\_type} function. 
The prototype of this function is given below
\begin{Code}
\begin{listing}[200]{2}
int STk_add_new_type(STk_extended_scheme_type *p)
\end{listing}
\end{Code}
The integer returned by this function is the (unique) key associated to the new
type. This key is stored in each cell of the new type.

We have now enough material to define the {\tt
STk\_extended\_scheme\_type} for the new type {\em stack}. This
declaration can be done in the following way:
\begin{Code}
\begin{listing}[200]{2}
static void mark_stack(SCM p);
static void free_stack(SCM p);
static void display_stack(SCM s, SCM port, int mode);
static int tc_stack;
\end{listing}
\end{Code}
\begin{Code}
\begin{listing}[200]{2}
static STk_extended_scheme_type stack_type = {
  "stack",              /* name */
  0,                    /* is_procp */
  mark_stack,           /* gc_mark_fct */
  free_stack,           /* gc_sweep_fct */
  NULL,                 /* apply_fct */
  display_stack         /* display_fct */
};
\end{listing}
\end{Code}

This definition tells the interpreter that the new type is not a
procedure (field  {\tt is\_procp} is set to 0). Consequently, the 
{\tt apply\_fct} is set to NULL. Note that a display function is
provided here. It permits to used a customized printing
function. 

\subsubsection{New type instances creation}

Creation of a new instance of the extended type necessitates the
definition of a constructor function. This constructor obeys always
the same framework. First you have to create a new cell with the
\Indextt{NEWCELL} macro. This macro has two parameters, a {\tt SCM}
object which will point the new cell and the type of the cell to
create. The second argument is generally equal to the value returned by 
\Indextt{STk\_add\_new\_type}. 
Once the cell is created, we have generally to (dynamically) allocate
a C structure which contains the informations which are necessary to
implement the new type. Dynamic allocation  can be done with the
function \Indextt{STk\_must\_malloc}. The area returned by 
\Indextt{STk\_must\_malloc} must be stored in the {\tt data} field of
the new cell. This field can be accessed with the \Indextt{EXTDATA}
macro. 

Let's go back to the stack example. We can now define a new primitive
function to make a new stack. Provided that the global variable {\tt tc\_stack}
already contains the value returned by {\tt STk\_add\_new\_type}, we
can write

\begin{Code}
\begin{listing}[200]{2}
#define STACKP(x)       (TYPEP(x, tc_stack))
#define NSTACKP(x)      (NTYPEP(x, tc_stack))
#define STACK(x)        ((Stack *) EXTDATA(x))

typedef struct {
  int len;
  SCM values;
} Stack;
\end{listing}
\end{Code}
\begin{Code}
\begin{listing}[200]{2}
static PRIMITIVE make_stack(void)
{
  SCM z;

  NEWCELL(z, tc_stack);
  EXTDATA(z)       = STk_must_malloc(sizeof(Stack));
  STACK(z)->len    = 0;
  STACK(z)->values = NIL;
  return z;
}
\end{listing}
\end{Code}

Here, the {\tt Stack} structure is used to represent a stack. This
structure contains two fields: {\tt len} and {\tt values}. Since the latter
field is a {\tt SCM} object, it must be recursively marked when a
stack is marked. We can now define the utility function necessary for
the GC:
\begin{Code}
\begin{listing}[200]{2}
static void mark_stack(SCM p)
{
  STk_gc_mark(STACK(p)->values);
}

static void free_stack(SCM p)
{
  free(EXTDATA(p));
}
\end{listing}
\end{Code}

To terminate with this example, we give below the code of the primitive 
{\tt stack\_push!}. Other primitive are built in the same fashion and
will not be described here. A complete listing of the stack
implementation is given in appendix.
\begin{Code}
\begin{listing}[200]{2}
static PRIMITIVE stack_push(SCM s, SCM val)
{
  Stack *sp;

  if (NSTACKP(s)) STk_err("stack-push: bad stack", s);

  sp         =  STACK(s);
  sp->len   += 1;
  sp->values = Cons(val, sp->values);
  
  return UNDEFINED;
}
\end{listing}
\end{Code}

\subsection{Definition of a C extended type}

The {\stk} interpreter permits to handle C pointers as first class
objects. 
[for eg: find an example to explain how it works: gdbm?]

\subsection{About memory: Common pitfalls}


\section{Loading an extension}

{\stk} support dynamic loading for several architectures/systems. The
way to provide dynamic loadable modules is different from one system
to another and you will have to adapt what is said here to the
conventions used by your system, architecture or compiler. 
Static loading can be used for systems which doesn' support dynamic
loading (such as Ultrix) or for which the interpreter doesn't support
yet dynamic loading.

\paragraph*{Note:} {\stk) also supports the \Index{DLD Gnu package} for
dynamic loading. DlD is a library package of C functions that performs
"dynamic link editing".  Since the time to load dynamically a module
with this package is rather long, it is preferred to avoid to use it.
However, this package is the only way to provide dynamic loading on
Linux systems which don't support the \Index{ELF} format (versions 1.0
to 1.2). Since the ELF format is becoming the new standard for Linux,
this package will be no more necessary in the future.

The last version of the DLD package can be found at several places:
{\tt
\begin{itemize}
\item ftp-swiss.ai.mit.edu:pub/scm
\item prep.ai.mit.edu:/pub/gnu/jacal
\item ftp.cs.indiana.edu:/pub/scheme-repository/imp/SCM-support
\end{itemize}
}

We suppose here that we want to include the {\tt posix} module defined
in section\ref{simple-example} into the {\stk} interpreter. 
\begin{Lentry}

\item [Dynamic Loading]
If the system running {\stk} supports dynamic loading (and if the
interpreter has been compiled with dynamic loading support), you
compile your source file file to make a {\em \Index{shared object}}
file. On SunOs~4.1, for instance, this can be done by compiling the
module with the {\em \Index{pic compilation option}} (pic stands for 
{\em position independent code}). Once compilation is done, you can
pre-load your file with the line
\begin{Code}
\begin{listing}[200]{2}
ld -assert pure-text -o time.so time.o
\end{listing}
\end{Code}
This will produce a file name {\tt posix.so} which can be loaded with the 
\Indextt{load} Scheme primitive procedure. The {\tt load} primitive recognizes
that this file is a shared object and calls a function whose
name the concatenation of the string 
{\tt STk\_init\_}\index{STk\_init\_ prefix} and the base name of file
loaded. Thus, loading the file {\tt posix.so} impies the call of a
pimitive whose name is {\tt STk\_init\_posix}. 

Look at {\tt Src/Extensions} directory to see some examples of shared
object construction.

\paragraph*{Note:} when the {\stk} is built, the \Indextt{Makefile}  in
the {\tt Src/Extensions} is customized for your system/compiler. A
simple way to determine the options you have to use for compiling your
program consist to run the \Indextt{make} command on one of the file present in
this directory. For instance, issuing the following command
\begin{Code}
\begin{listing}[200]{2}
make -n posix.so
\end{listing}
\end{Code}
on a Linux box using the DLD package will output the following lines:
\begin{Code}
\begin{listing}[200]{2}
gcc  -g -DSTk_CODE -DUSE_DLD -DLINUX   -DHAVE_UNISTD_H=1 \
        -DHAVE_SIGACTION=1 -I../Tk -I../Tcl -I../Src -I../Mp \
        -I/usr/X11R6/include  -c posix.c -o posix.o
ld -r -o posix.so posix.o
\end{listing}
\end{Code}

\item [Static loading]
A C module which define a new type can also be statically loded in the
interepreter. To load your module, you have to modify the {\tt Src/Makefile}
or {\tt  Snow/Makefile}. Once yo have added your extension object in
the \Indextt{USER\_OBJ} variable , you must modify the file
\Indextt{Src/userinit.c} to add you initialization (and eventually
cleanup) code. The call to your initialization function must be done in
the \Indextt{STk\_user\_init} C function. Once this is done, you can
run the {\tt make} command again to build the extended interpreter.
\end{Lentry}

\section{Adding new Tk widgets}

\subsection{Widget compilation}

Adding a new Tk widget to the {\stk} interpreter is generally a simple
{\em hack}. Most of the time, extension widgets written
for Tcl/Tk can be added to the {\stk} interpreter without modifying
the source code of the widget. However, there is no unique method to
add a widget to the Tcl interpreter; consequently, what is given below
is a set of hints to widget integration rather than a {\em always
working recipe}. To illustrate this section, we will see how we can add
the {\em fscale} widget (a floating-point scale widget available on the
Tcl/Tk repository in the {\tt tkFScale-?.?.tar.gz} file)\footnote{
This widget is now integrated in the standard Tk4.0; it is provided as
an extension widget with Tk3.6.}.

Generally, the code of a Tcl/Tk extension widget can be divided in two
parts: the code which implement widget's behavior and the extension
initialization code. Extension initialization code, in Tcl/Tk, must be
placed in the procedure {\tt Tcl\_AppInit} which is located in the file
{\tt tkAppInit.c}. If the extension package adds a lot of widgets, it
generally defines a function to do all the initializations. On the
other hand, if the extension only defines a single widget, the
extension code generally consists to call the C function 
\Indextt{Tcl\_CreateCommand} for each new widget defined in the
extension. {\tt Tcl\_CreateCommand} is the Tcl standard way to add a
new command. This function also exists in the {\stk} interpreter; it
creates a new \Index{Tk command} object~\cite{Gallesio95-1}. The
prototype of this function is:
\begin{Code}
\begin{listing}[200]{2}
void Tcl_CreateCommand(Tcl_Interp *interp,
                       char *cmdName, 
                       Tcl_CmdProc *proc,
                       ClientData clientData,
                       Tcl_CmdDeleteProc *deleteProc));
\end{listing}
\end{Code}

For {\stk},
\begin{itemize}
\item the {\tt interp} is always the global variable
                \Indextt{Stk\_main\_interp}
\item {\tt cmdName} is the name of the widget in the Scheme world.
\item {\tt proc} is the name of the C function which implement the {\em Tk command}
\item {\tt clientData} are informations which are associated to the
widget code. For a new widget, {\tt clientData} can generally set to 
the result of 
\begin{Code}
\begin{listing}[200]{2}
Tk_MainWindow(Stk_main_interp);
\end{listing}
\end{Code}
\item {\tt deleteProc} is a function which is called when the widget
is destroyed. You generally don't need to change the value of this
parameter (which is often set to {\tt NULL}). 
\end{itemize}

The usual way to integrate this initialization code in a Tcl interpreter
consists to patch the {\tt tkAppInit.c} file to add the call to the
initialization (or the {\tt Tcl\_CreateCommand}) function.
To add an extension written for Tcl/Tk to {\stk}, all that is needed
consists to adapt the initialization code for {\stk}. For example, the
{\tt fscale} widget initialization code adds the following call in the
body of the {\tt Tcl\_AppInit} function:
\begin{Code}
\begin{listing}[200]{2}
Tcl_CreateCommand(interp, "fscale", Tk_FScaleCmd, 
                  (ClientData) main, 
                  (void (*)()) NULL);
\end{listing}
\end{Code}

For {\stk}, this call can be written 
\begin{Code}
\begin{listing}[200]{2}
Tcl_CreateCommand(STk_main_interp, "fscale", Tk_FScaleCmd, 
                  (ClientData) Tk_MainWindow(STk_main_interp), 
                  (void (*)()) NULL);
\end{listing}
\end{Code}

This call must executed before trying to create a new {\tt fscale}
widget.

\subsection{Widget linking}

The cleaner way to add a new widget to {\stk} consists to define a
special C module for this widget. Defining the widget in a C module
allows us to make the new widget dynamically loadable. The code for 
making the {\tt fscale} widget dynamically loadable could be:

\begin{Code}
\begin{listing}[200]{2}
/* Contents of the file fscale.c */
#ifndef USE_TK
#define USE_TK
#endif
\end{listing}
\end{Code}
\begin{Code}
\begin{listing}[200]{2}
#include <stk.h>
/* 
 *include the widget source code. Ugly but this avoid to have two
 * source files to link
 */
#include "tkFScale.c"
\end{listing}
\end{Code}
\begin{Code}
\begin{listing}[200]{2}
PRIMITIVE STk_init_fscale(void)
{
  Tcl_CreateCommand(STk_main_interp, 
                    "fscale", 
                    Tk_FScaleCmd, 
                    (ClientData) Tk_MainWindow(STk_main_interp), 
                    (void (*)()) NULL);
}
\end{listing}
\end{Code}

\section{Extending the interpreter with C++}

[for eg : Identify the problems]

\section{Embedding the STk interpreter}

[for eg: This parts need some work in the interpreter]



%%%%%%%%%%%%%%%%%%%%%%%%%%%%%%%%%%%%%%%%%%%%%%%%%%%%%%%%%%%%%%%%%%%%%%%%%%%%%%

\pagebreak
{\Large\bf Appendix}
\par
Hereafter is the complete code for the stack type discussed in \ref{stack}
{\footnotesize
\verbatiminput{stack.c}
}

%%%%%%%%%%%%%%%%%%%%%%%%%%%%%%%%%%%%%%%%%%%%%%%%%%%%%%%%%%%%%%%%%%%%%%%%%%%%%%
\bibliography{bibliography}
\input{Extending.ind}

\end{document}
% LocalWords:  STk RS unsrt pt Erick Gallesio Universit de Antipolis CNRS URA
% LocalWords:  Laboratoire ESSI des Colles Cedex mm email eg unice fr fancybox
% LocalWords:  fancyheadings fancyplain posix ctime UTC makeinteger tc subr SCM
% LocalWords:  arity makestring stk stklos scm init sec struct obj Src int MIN
% LocalWords:  lsubr argcount fsubr env tkcommand Tcl argc argv makevect CDR GC
% LocalWords:  VECT NULLP eval Ntruth LISTx val foo var errno resp pre gc fct
% LocalWords:  args EXT ISPROC EVALPARAM getc ungetc putc eof DSP WRT TK Tk pic
% LocalWords:  Solaris SunOs ld


\end{document}
% LocalWords:  STk RS unsrt pt Erick Gallesio Universit de Antipolis CNRS URA
% LocalWords:  Laboratoire ESSI des Colles Cedex mm email eg unice fr fancybox
% LocalWords:  fancyheadings fancyplain posix ctime UTC makeinteger tc subr SCM
% LocalWords:  arity makestring stk stklos scm init sec struct obj Src int MIN
% LocalWords:  lsubr argcount fsubr env tkcommand Tcl argc argv makevect CDR GC
% LocalWords:  VECT NULLP eval Ntruth LISTx val foo var errno resp pre gc fct
% LocalWords:  args EXT ISPROC EVALPARAM getc ungetc putc eof DSP WRT TK Tk pic
% LocalWords:  Solaris SunOs ld


\end{document}
% LocalWords:  STk RS unsrt pt Erick Gallesio Universit de Antipolis CNRS URA
% LocalWords:  Laboratoire ESSI des Colles Cedex mm email eg unice fr fancybox
% LocalWords:  fancyheadings fancyplain posix ctime UTC makeinteger tc subr SCM
% LocalWords:  arity makestring stk stklos scm init sec struct obj Src int MIN
% LocalWords:  lsubr argcount fsubr env tkcommand Tcl argc argv makevect CDR GC
% LocalWords:  VECT NULLP eval Ntruth LISTx val foo var errno resp pre gc fct
% LocalWords:  args EXT ISPROC EVALPARAM getc ungetc putc eof DSP WRT TK Tk pic
% LocalWords:  Solaris SunOs ld


\end{document}
% LocalWords:  STk RS unsrt pt Erick Gallesio Universit de Antipolis CNRS URA
% LocalWords:  Laboratoire ESSI des Colles Cedex mm email eg unice fr fancybox
% LocalWords:  fancyheadings fancyplain posix ctime UTC makeinteger tc subr SCM
% LocalWords:  arity makestring stk stklos scm init sec struct obj Src int MIN
% LocalWords:  lsubr argcount fsubr env tkcommand Tcl argc argv makevect CDR GC
% LocalWords:  VECT NULLP eval Ntruth LISTx val foo var errno resp pre gc fct
% LocalWords:  args EXT ISPROC EVALPARAM getc ungetc putc eof DSP WRT TK Tk pic
% LocalWords:  Solaris SunOs ld
